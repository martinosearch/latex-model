\documentclass[11pt]{article}
\usepackage{xltxtra}
\usepackage{xgreek}
\usepackage{amsmath,amssymb,stmaryrd,calc}%}% pour geqslant qui existe ds fourier
\usepackage{xkeyval}
\usepackage{multirow,longtable}
\usepackage[%
      a4paper,%
      textwidth=16cm,
      top=2cm,%
      bottom=2cm,%
      headheight=25pt,%
      headsep=12pt,%
      footskip=25pt]{geometry}%
\usepackage[greek]{alterqcm}
\usepackage{tikz}
%%%%%%%%%%%%%%%%%%%%%%%%%%%%%%%%%%%%%%%%%%%%%%%%%%%%%%%%%%%%%%


%%%%%%%%%%%%%%%%%%%%%%%%%%%%%%%%%%%%%%%%%%%%%%%%%%%%%%%%%%%%%%
\parindent=0pt
\begin{document}
\setmainfont[Mapping=tex-text,Ligatures=Common]{Minion Pro}

%%%%%%%%%%%%%%%%%%%%%%%%%%%%%%%%%%%%%%%%%%%%%%%%%%%%%%%
%\nogreekalph 
\begin{minipage}[t][][b]{.45\linewidth}
	Έστω $f$ ορισμένη και παραγωγίσιμη στο διάστημα $\big[-3,\,+\infty\big)$,
	αύξουσα στα διαστήματα $\big[-3,\,-1\big]$ et $\big[2,\,+\infty\big)$
	και φθίνουσα στο διάστημα $\big[-1,\,2\big]$.
Έστω $f^{\prime}$ η παράγωγός της στο διάστημα $[-3,\,+\infty)$.
	Η γραφική παράσταση $\Gamma$ της $f$ είναι σχεδιασμένη στο διπλανό σχήμα ως προς ένα ορθογώνιο σύστημα αξόνων $\big(O,~\vec{\imath},~\vec{\jmath}\big)$.
	Διέρχεται από το σημείο A$(-3,\,0)$ και δέχεται ως ασύμπτωτη της ευθεία
	$(\delta)$ με εξίσωση $y = 2x -5$.
\end{minipage}
\begin{minipage}[t][][b]{.45\linewidth}
	\null
	\begin{tikzpicture}[scale=0.5,>=latex]
	\draw[very thin,color=gray] (-3,-2) grid (10,8);
	\draw[->] (-3,0) -- (10,0) node[above left] {\small $x$};
	\foreach \x in {-3,-2,-1,1,2,...,9}
	\draw[shift={(\x,0)}] (0pt,1pt) -- (0pt,-1pt)node[below] { $\x$};
	\draw[->] (0,-2) -- (0,8) node[below right] {\small $y$};
	\foreach \y/\ytext in {-2,-1,1,2,...,8}
	\draw[shift={(0,\y)}] (1pt,0pt) -- (-1pt,0pt) node[left] { $\y$};
	\draw (-0.5,-2) -- (10,8);
	\node[above right] at (-3,0) {\textbf{A}};
	\node[above right] at (0,0) {\textbf{O}};
	\node[below right] at (4,3) {$\mathbf{\Delta}$};
	\node[above right] at (4,5) {$\mathbf{\Gamma}$};
	\draw plot[smooth] coordinates{%
		(-3,0)(-2,4.5)(-1,6.5)(0,5.5)(1,3.5)(2,3)(3,3.4)(4,4.5)(5,6)(6,7.75)};
	\end{tikzpicture}
\end{minipage}

\vspace{20pt}
\begin{alterqcm}[VF,pre=true,lq=125mm]
	\AQquestion{Για κάθε $x \in (-\infty,\,2],\;f^{\prime}(x) \geqslant 0$.}
	\AQquestion{Η συνάρτηση $F$ παρουσιάζει μέγιστο στο $2$}
	\AQquestion{$\displaystyle\int_{0}^2 f’(x)\:\text{d}x = - 2$}
\end{alterqcm}
%\greekalph    %%% <----------------------------------------------

\end{document}
