% \iffalse meta-comment
%
%
% This file is part of the Standard LaTeX `Tools Bundle'.
% -------------------------------------------------------
%
% It may be distributed and/or modified under the
% conditions of the LaTeX Project Public License, either version 1.3c
% of this license or (at your option) any later version.
% The latest version of this license is in
%    https://www.latex-project.org/lppl.txt
% and version 1.3c or later is part of all distributions of LaTeX
% version 2005/12/01 or later.
%
% The list of all files belonging to the LaTeX `Tools Bundle' is
% given in the file `manifest.txt'.
%
% \fi
%
% \iffalse    This is a METACOMMENT
%
%% Package `multicol' to use with LaTeX2e
%% Copyright 1989-2018 Frank Mittelbach
%%
%%  In addition to the terms of LPPL any distributed version
%%  (unchanged or modified) of multicol has to keep the statement
%%  about the moral obligation for using multicol. In case of major
%%  changes where this would not be appropriate the author of the
%%  changed version should contact the copyright holder.
%%
%%
%%  Moral obligation for using multicol:
%%  ------------------------------------
%%
%%  Users of multicol who wish to include or use multicol or a modified
%%  version in a proprietary and commercially market product are asked
%%  under certain conditions (see below) for the payment of a license
%%  fee.  The size of this fee is to be determined, in each instance,
%%  by the commercial user, depending on his/her judgment of the value of
%%  multicol for his/her product.
%%
%%
%%  The conditions for this are as follows:
%%
%%   The producer of a proprietary and commercially market product
%%   that involves typesetting using multicol is asked to determine
%%   the value of a license fee for using multicol if
%%
%%   - the product is a document and the producer has decided to
%%     include multicol to typeset (parts of) the document or has
%%     directed the author of the document to include multicol (for
%%     example, by providing a class file to be used by the author)
%%
%%   - the product is a LaTeX class or package that includes multicol
%%
%%
%%   There is no moral obligation in case
%%
%%   - the product is a document but producer has not directed
%%     the author to include multicol (in that case the moral obligation
%%     lies with the author of the document)
%%
%%   - the product does not involve typesetting, e.g., consists, for
%%     example, of distributing multicol and its documentation.
%%
%%   - the product is not proprietary, i.e., is made available as free
%%     software itself (which doesn't prohibit its commercial marketing)
%%
%%   - multicol is used for non-commercial purposes
%%
%%
%% Determining a license fee might result in a license fee of zero
%% (i.e., no payment) in case a producer has determined that the use
%% of multicol has no enhancing effect on the product. This is a
%% plausible scenario, i.e., in the above two cases the producer is
%% only asked to evaluate the value of multicol for the product
%% not for the payment of a license fee per se (which might or might
%% not follow from this evaluation).
%%
%% The license fee, if any, can be payed either to the LaTeX3 fund
%% (see ltx3info.txt in the base LaTeX distribution) or to the author of
%% the program who can be contacted at
%%
%%     Frank.Mittelbach@latex-project.org
%%
%<*dtx>
          \ProvidesFile{multicol.dtx}
%</dtx>
%<package>\NeedsTeXFormat{LaTeX2e}[2018-04-01]
%<package>
%<package>\providecommand\DeclareRelease[3]{}
%<package>\providecommand\DeclareCurrentRelease[2]{}
%<package>
%<package>\DeclareRelease{}{2017-04-11}{multicol-2017-04-11.sty}
%<package>\DeclareCurrentRelease{}{2018-04-01}
%<package>
%<package>\ProvidesPackage{multicol}
%<driver> \ProvidesFile{multicol.drv}
% \fi
%         \ProvidesFile{multicol.dtx}
          [2018/04/20 v1.8s  multicolumn formatting (FMi)]
%
%
%%
%
% \changes{v1.5n}{1997/06/05}{Applied improvement of documentation,
%          kindly done by Robin Fairbairns.}
% \changes{v1.4h}{1992/06/04}{Added mark tracing with
%                           tracingmulticols$\ge2$}
% \changes{v1.4a}{1992/02/11}{Added support for multicol in inner mode}
% \changes{v1.0d}{1989/05/17}{All lines shortened to 72 or less.}
% \changes{v1.0e}{1989/06/21}{Redefinition of description env. to use
%                           \cs{descriptionmargin}\quotechar=5pt
%                           in documentation.}
% \changes{v1.0f}{1989/07/11}{Changed \cs{z@} to 0pt in redefinition of
%                           description.}
% \changes{v1.1a}{1989/09/20}{\cs{multicolssep} changed to \cs{multicolsep}.}
%
% \def\description{\list{}{\labelwidth 0pt \leftmargin\descriptionmargin
%    \itemindent-\leftmargin \let\makelabel\descriptionlabel}}
% \newdimen\descriptionmargin \descriptionmargin=5pt
%
% \DoNotIndex{\@M,\@Mi,\@bsphack,\@cclv,\@colht,\@currlist,\@deferlist}
% \DoNotIndex{\@elt,\@esphack,\@floatplacement}
% \DoNotIndex{\@ifundefined,\@ifnextchar,\@makecol}
% \DoNotIndex{\@ne,\@outputpage,\@scolelt,\@spaces,\@tempb,\@tempcnta}
% \DoNotIndex{\@width}
% \DoNotIndex{\addvspace,\advance,\allowbreak}
% \DoNotIndex{\baselineskip,\begingroup,\box,\columnsep,\copy,\count}
% \DoNotIndex{\count@,\def,\dimen@,\divide,\docdate}
% \DoNotIndex{\edef,\eject,\egroup,\else,\endgroup,\endinput}
% \DoNotIndex{\fi,\fileversion,\filedate}
% \DoNotIndex{\gdef,\global}
% \DoNotIndex{\hbox,\hfil,\hrule,\ht,\hss}
% \DoNotIndex{\ifdim,\ifnum,\ifvoid,\ignorespaces,\insert,\immediate}
% \DoNotIndex{\let,\loop}
% \DoNotIndex{\maxdepth,\message,\multiply}
% \DoNotIndex{\newbox,\newcount,\newdimen,\newskip,\number,\newpage}
% \DoNotIndex{\outputpenalty,\p@,\penalty}
% \DoNotIndex{\relax,\repeat,\setbox,\skip,\space,\splitmaxdepth}
% \DoNotIndex{\splittopskip,\string,\sixt@@n}
% \DoNotIndex{\the,\thepage,\thr@@,\topskip,\tw@,\typeout}
% \DoNotIndex{\unvbox,\vbox,\vfill,\vsplit,\voidb@x,\vrule}
% \DoNotIndex{\write,\wd}
% \DoNotIndex{\z@}
%
% \MakeShortVerb{\|}
% \newcommand{\mc}{{\sf multicols}}
% \newcommand{\TUB}{{\sl TUGboat\/}}
% \newcommand{\TB}{{\sl\TeX book\/}}
%
%
% \setcounter{StandardModuleDepth}{2}
% \setcounter{collectmore}{3}
%
% \GetFileInfo{multicol.dtx}
% \title{An environment for multicolumn output\thanks{This file
%        has version number \fileversion, last
%        revised \filedate.}%
%       \thanks{Note: This package is released under terms which affect
%       its use in commercial applications. Please see the details at
%       the top of the source file.}}
% \author{Frank Mittelbach\\
%       {\rm Email:} see top of the source file}
% \date{Printed \today}
%
% \MaintainedByLaTeXTeam{tools}
% \maketitle
%
% \begin{abstract}
%   This article describes the use and the implementation of the \mc{}
%   environment. This environment allows switching between
%   one and multicolumn format on the same page. Footnotes are handled
%   correctly (for the most part), but will be placed at the bottom of
%   the page and not under each column.  \LaTeX{}'s float mechanism,
%   however, is partly disabled in this implementation.  At the
%   moment only page-wide floats (i.e., star-forms) can be used within
%   the scope of the environment.
% \end{abstract}
%
% \begin{multicols}{3}[\section*{Preface to version 1.8}]
%    The 1.8 release improves on the balancing approach. If due to a
%    limited number of break points (e.g., due to large objects) the
%    balanced columns exceed the available vertical space, then
%    balancing is canceled and a normal page is produced first. Some
%    overflow is allowed (controlled by the parameter
%    |\maxbalancingoverflow| which defaults to 12pt). This ensures
%    that we only cut a normal page if we get enough material carried
%    over to next page.
%
%    Also added was support for |\enlargethispage|. This means it is
%    now possible to request a page to be artificially  enlarged or
%    shortened. Note that if you enlarge pages by more than one line
%    you may have to increase the {\sf collectmore} counter value to
%    ensure that enough material is being picked up.
%
%    This command was used on the second page of this manual to shorten
%    it by one line, in order to get rid of a number of widow lines on
%    the following pages.
%
%
%    Finally, version 1.8 adds the command |\docolaction| to help with
%   more complicated actions that depend on the current column. This
%   command expects 3 arguments: code that is executed if we are in
%   the ``first'' column, code to execute if we end up in any
%   ``middle'' column (if there are more than two) and finally code to
%   execute if we are in the ``last'' column. Thus
%\begin{verbatim}
%  \docolaction{first}
%              {middle}{last}
%\end{verbatim}
%   would typeset a different word depending the type of column this
%   code is executed. Using it like this is probably pointless, but
%   you can imagine applications like writing something into the
%   nearest margin, etc.
%
%   As this feature needs at least two \LaTeX{} runs to produce correct
%   results and as it adds to the processing complexity it is only
%   made available if one add the option \texttt{colaction} when
%   loading the package.
%
% \end{multicols}
%
%
% \begin{multicols}{3}[\section*{Preface to version 1.7
%                         (right to left support)}]
%    The 1.7 release adds support for languages that are typeset
%    right-to-left. For those languages the order of the columns on the page
%    also need to be reversed---something that wasn't supported before. The next
%    paragraph demonstrates the result (as it is typeset as if we are writing
%    in a left-to-right language--- so read the rightmost column
%    first). The change is initialized via
%    |\RLmulticolcolumns| and returning to left-right (default)
%    is done via |\LRmulticolcolumns|.
% \end{multicols}
%
% \RLmulticolcolumns
%
% \begin{multicols}{3}
%    Right-to-left typesetting will only reverse the column orders. Any
%    other support needed will have to be provided by other means,
%    e.g., using appropriate fonts and reversing the writing directions
%    within the columns. As footnotes are typeset in full measure the
%    footnote rule needs to be redefined as if they are below a single
%    column, i.e., using |\textwidth| not |\columnwidth|. For example:
%\begin{verbatim}
%\renewcommand \footnoterule{%
%  \kern-3pt\hbox to\textwidth
%    {\hskip .6\textwidth
%     \hrulefill }%
%  \kern2.6pt}
%\end{verbatim}
% \end{multicols}
%
% \LRmulticolcolumns
%
% \begin{multicols}{3}[\section*{Preface to version 1.5 + 1.6}]
%   \hbadness=10000
%   The 1.5 release contains two major changes: \mc{} will now
%   support up to 10 columns and two more tuning possibilities have
%   been added to the balancing routine. The balancing routine now
%   checks the badness
%   of the resulting columns and rejects solutions that are larger
%   than a certain threshold.
%   At the same time \mc{} has been upgraded to run under \LaTeXe{}.
%
%   Later changes to 1.5 include |\columnbreak| and |multicols*|.
%
%   For version 1.6 micro-spacing around the boxes produced by \mc{}
%   has been improved to allow for baseline-grid typesetting.
% \end{multicols}
%
%
% \setcounter{collectmore}{2}
% \begin{multicols}{3}[\section{Introduction}]
%   \hbadness=10000
%   Switching between two-column and one-column layout is possible in
%   \LaTeX{}, but every use of |\twocolumn| or |\onecolumn|
%   starts a new page. Moreover, the last page of two-column output
%   isn't balanced and this often results in an empty, or nearly
%   empty, right column. When I started to write macros for {\sf
%   doc.sty} (see ``The {\tt doc}--Option'', \TUB\
%   volume 10~\#2, pp.~245--273) I thought that it would be nice to
%   place the index
%   on the same page as the bibliography. And balancing the last page
%   would not only look better, it also would save space; provided of
%   course that it is also possible to start the next article on the
%   same page. Rewriting the index environment was comparatively easy,
%   but the next goal, designing an environment which takes care of
%   footnotes, floats, etc., was a harder task. It took me a whole
%   weekend\footnote{I started with the algorithm given in the \TeX
%   book on page 417. Without this help a weekend would not have been
%   enough. (This remark was made in the documentation of the initial
%    release, since then several hundreds more hours went into
%    improving the original code.)} to get together the few lines of
%    code below and there is
%   still a good chance that I missed something after all.
%
%   Try it and, hopefully, enjoy it; and {\em please\/} direct bug
%   reports and suggestions back to Mainz.
% \end{multicols}
%
%
% \setcounter{collectmore}{0}
% \begin{multicols}{3}[\section{The User Interface}]
%   \hbadness=10000
%   To use the environment one simply says\\*[2mm]
%     \hspace*{2mm}|\begin{multicols}{|\meta{number}|}|
%     \hspace*{12mm}\meta{multicolumn text}\\
%     \hspace*{2mm}|\end{multicols}|\\[2mm]
%   where \meta{number} is the required number of columns and
%^^A\meta{multicolumn text}
%   $\langle${\it multi\-column text\/}$\rangle$ may contain arbitrary
%   \LaTeX{} commands, except that floats and marginpars are not
%   allowed in the current implementation\footnote{This is dictated by
%   lack of time.  To implement floats one has to reimplement the
%   whole \LaTeX{} output routine.}.
%
% \enlargethispage{-1\baselineskip}
%
% \DescribeMacro\premulticols
% As its first action, the {\sf multicols} environment measures the
% current page to determine whether there is enough room for some
% portion of multicolumn output. This is controlled by the
% \meta{dimen} variable |\premulticols| which can be changed by
% the user with ordinary \LaTeX{} commands.
% \DescribeMacro\multicolsep
% If the space is less than |\premulticols|, a new page is
% started.  Otherwise, a |\vskip| of |\multicolsep| is
% added.\footnote{Actually the added space may be less because we use
% \cs{addvspace} (see the \LaTeX{} manual for further
% information about this command).}
%
% \DescribeMacro\postmulticols
% When the end of the \mc{} environment is encountered, an
% analogous mechanism is employed, but now we test whether there is a
% space larger than |\postmulticols| available. Again we add
% |\multicolsep| or start a new page.
%
% It is often convenient to spread some text over all columns, just
% before the multicolumn output, without any page break in between. To
% achieve this the \mc{} environment has an optional second
% argument which can be used for this purpose. For example, the text
% you are now reading was started with
% \begin{verbatim}
% \begin{multicols}{3}
%   [\section{The User
%             Interface}] ...
%\end{verbatim}
% If such text is unusually long (or short) the value of
% |\premulticols| might need adjusting to prevent a bad page
% break.  We therefore provide a third argument which can be used to
% overwrite the default value of |\premulticols| just for this
% occasion. So if you want to combine some longer single column text
% with a multicols environment you could write
% \begin{verbatim}
% \begin{multicols}{3}
%     [\section{Index}
%      This index contains ...]
%     [6cm]
%  ...
%\end{verbatim}
%
%
% \DescribeMacro\columnsep
% The space between columns is controlled by the length parameter
% |\columnsep|. The width for the individual columns is automatically
% calculated from this parameter and the current |\linewidth|. In this
% article a value of \textsf{\the\columnsep} was used.
%
%
% \DescribeMacro\columnseprule
% Separation of columns with vertical rules is achieved by setting the
% parameter |\columnseprule| to some positive value.  In this
% article a value of {\sf.4pt} was used.
%
% \DescribeMacro\columnseprulecolor
% The color of the rules separating the columns can be specified through
% |\columnseprulecolor|. The default value is |\normalcolor|.
%
% \DescribeMacro\multicolbaselineskip
% Since narrow columns tend to need adjustments in interline spacing
% we also provide a \meta{skip} parameter called
% |\multicolbaselineskip| which is added to the
% |\baselineskip| parameter inside the \mc{}
% environment.  Please use this parameter with care or leave it alone;
% it is intended only for package file designers since even small
% changes might produce totally unexpected changes to your document.
%
%
%    \subsection{Balancing columns}
%
%
% Besides the previously mentioned parameters, some others are
% provided to influence the layout of the columns generated.
%
% Paragraphing in \TeX{} is controlled by several parameters. One of
% the most important is called |\tolerance|: this controls the
% allowed `looseness' (i.e.\ the amount of blank space between words).
% Its default value is 200 (the \LaTeX{} |\fussy|) which is too
% small for narrow columns. On the other hand the |\sloppy|
% declaration (which sets |\tolerance| to $10000=\infty$) is too
% large, allowing really bad spacing.\footnote{Look at the next
% paragraph, it was set with the \cs{sloppy} declaration.}
%
% \begin{sloppypar}
% \DescribeMacro\multicoltolerance \DescribeMacro\multicolpretolerance
%  We therefore use a |\multicoltolerance| parameter for the
% |\tolerance| value inside the \mc{} environment.  Its default value
% is 9999 which is less than infinity but `bad' enough for most
% paragraphs in a multicolumn environment. Changing its value should
% be done outside the \mc{} environment.  Since |\tolerance| is set
% to |\multicoltolerance| at the beginning of every {\sf multicols}
% environment one can locally overwrite this default by assigning
% \verb*+\tolerance = +\meta{desired value}. There also exists a
% |\multicolpretolerance| parameter holding the value for
% |\pretolerance| within a \mc{} environment. Both parameters are
% usually used only by package designers.
% \end{sloppypar}
%
% Generation of multicolumn output can be divided into two parts.  In
% the first part we are collecting material for a page, shipping it
% out, collecting material for the next page, and so on.  As a second
% step, balancing will be done when the end of the \mc{} environment
% is reached.
% In the first step \TeX{} might consider more material whilst
% finding the final column content than it actually uses when shipping out the
% page. This might cause a problem if a footnote is encountered in
% the part of the input considered, but not used, on the current page.
% In this case the footnote might show up on the current page, while the
% footnotemark corresponding to this footnote might be set on the next
% one.\footnote{The reason behind this behavior is the asynchronous
%               character of the \TeX{} {\it page\_builder}.
%               However, this
%               could be avoided by defining very complicated output
%               routines which don't use \TeX{} primitives like
%               \cs{insert} but do everything by hand.
%               This is clearly beyond the scope of a weekend problem.}
% Therefore the \mc{} environment gives a warning
% message\footnote{This message will be generated even if there are no
% footnotes in this part of the text.} whenever it is unable to use
% all the material considered so far.
%
% If you don't use footnotes too often the chances of something
% actually going wrong are very slim, but if this happens you can help
% \TeX{} by using a |\pagebreak| command in the final document.
% Another way to influence the behavior of \TeX{} in this respect is
% given by the counter variable `{\sf collectmore}'. If you use the
% |\setcounter| declaration to set this counter to \meta{number},
% \TeX{} will consider \meta{number} more (or less) lines before
% making its final decision. So a value of $-1$ may solve all your
% problems at the cost of slightly less optimal columns.
%
%
% In the second step (balancing columns) we have other bells and
% whistles. First of all you can say |\raggedcolumns| if you
% don't want the bottom lines to be aligned.
% The default is |\flushcolumns|, so \TeX{} will normally try to
% make both
% the top and bottom baselines of all columns align.
%
% Additionally you can set another counter, the `{\sf unbalance}'
% counter, to some positive \meta{number}. This will make all but the
% right-most column \meta{number} of lines longer than they would
% normally have been. `Lines' in this context refer to normal text
% lines (i.e.\ one |\baselineskip| apart); thus, if your columns
% contain displays,
% for example, you may need a higher \meta{number}
% to shift something from one column into another.
%
% Unlike `{\sf collectmore},' the `{\sf unbalance}' counter is reset
% to zero at the end of the environment so it only applies to one
% \mc{} environment.
%
% The two methods may be combined but I suggest using these features
% only when fine tuning important publications.
%
% Two more general tuning possibilities were added with version~1.5.
% \TeX{} allows to measure the badness of a column in terms of an
% integer value, where 0 means optimal and any higher value means a
% certain amount of extra white space. 10000 is considered to be
% infinitely bad (\TeX{} does not distinguish any further). In addition
% the special value 100000 means overfull (i.e., the column contains
% more text than could possibly fit into it).
%
% The new release now measures every generated column and ignores
% solutions where at least one column has a badness being larger than
% the value of the counter {\sf columnbadness}. The default value for
% this counter is 10000, thus \TeX{} will accept all solutions except
% those being overfull.
% By setting the counter to a smaller value you can force the algorithm
% to search for solutions that do not have columns with a lot of white
% space.
%
% However, if the setting is too low, the algorithm may not find any
% acceptable solution at all and will then finally choose the extreme
% solution of placing all text into the first column.
%
% Often, when columns are balanced, it is impossible to find a solution
% that distributes the text evenly over all columns. If that is the case
% the last column usually has less text than the others. In the earlier
% releases this text was stretched to produce a column with the same
% height as all others, sometimes resulting in really ugly looking
% columns.
%
% In the new release this stretching is only done if the badness of
% the final column is not larger than the value of the counter
% {\sf finalcolumnbadness}. The default setting is 9999, thus preventing
% the stretching for all columns that \TeX{} would consider infinitely
% bad. In that case the final column is allowed to run short which gives
% a much better result.
%
% And there are two more parameters of some experimental nature, one
% called |\multicolovershoot| the other |\multicolundershoot|.  They
% control the amount of space a column within the \mc{} environment is
% allowed to be ``too full'' or ``too short'' without affecting the
% column badness. They are set to 0pt and 2pt, respectively.
%
% Finally, when doing the balancing at the end, columns may become
% higher than the remaining available space. In that case the
% algorithm aborts and instead generates a normal page. However, if the
% amount is not too large, e.g., a line or so, then it might be better to
% keep everything on the same page instead of starting a new page with
% just one line after balancing. So the parameter
% |\maxbalancingoverflow| governs this process: only when the excess
% gets larger than its value balancing is aborted.
%
% \subsection{Not balancing the columns}
%
% Although this package was written to solve the problem of balancing
% columns, I got repeated requests to provide a version where all
% white space is automatically placed in the last column or
% columns. Since version v1.5q this now exists: if you use
% \texttt{multicols*} instead of the usual environment the columns on
% the last page are not  balanced. Of course, this environment only
% works on top-level, e.g., inside a box one has to balance to
% determine a column height in absence of a fixed value.
%
%
% \subsection{Manually breaking columns}
%
% Another request often voiced was: ``How do I tell \LaTeX{} that it
% should break the first column after this particular line?''. The
% |\pagebreak| command (which works with the two-column option of
% \LaTeX{}) is of no use here since it would end the collection phase
% of \mc{} and thus all columns on that page. So with version 1.5u
% the |\columnbreak| command was added. If used within a paragraph it
% marks the end of the current line as the desired breakpoint. You can
% observe its effect on
% the previous page where three lines of text have been artificially
% forced into the second column  (resulting in some white
% space between paragraphs in the first column).
%
% \subsection{Floats inside a \mc{} environment}
%
% Within the \mc{} environment the usual star float commands are
% available but their function is somewhat different as in the
% two-column mode of standard \LaTeX. Stared floats, e.g., {\tt
% figure*}, denote page wide floats that are handled in a similar
% fashion as normal floats outside the \mc{} environment. However,
% they will never show up on the page where they are encountered. In
% other words, one can influence their placement by specifying a
% combination of {\tt t}, {\tt b}, and/or {\tt p} in their optional
% argument, but {\tt h} doesn't work because the first possible place
% is the top of the next page. One should also note, that this means
% that their placement behavior is determined by the values of
% |\topfraction|, etc.\ rather than by |\dbl...|.
%
%
% \subsection{Support for right-to-left typesetting}
%
%    In right-to-left typesetting the order of the columns on the page
%    also need to be reversed, i.e., the first column has to appear on the far
%    right and the last column on the left. This is supported through the
%    commands |\RLmulticolcolumns| (switching to right-to-left typesetting)
%    and |\LRmulticolcolumns| (switching to left-to-right typesetting) the
%    latter being the default.
%
% \subsection{Warnings}
%
% Under certain circumstances the use of the \mc{} environment may
% result in some
% warnings from \TeX{} or \LaTeX{}. Here is a list
% of the important ones and the possible cause:
% \begin{description}
%
% \item[] {\hspace*{-\labelsep}\tt Underfull \string\hbox\space
%  (badness ...)}
%
%  As the columns are often very narrow \TeX{} wasn't able to find a
%  good way to break the paragraph. Underfull denotes a loose line but
%  as long as the badness value is below $10000$ the result is probably
%  acceptable.
%
% \item[]
% {\hspace*{-\labelsep}\tt Underfull \string\vbox\space ...  while
%  \string\output\space is active}
%
%  If a column contains a character with an unusual depth, for
%  example a `(', in the bottom line then this message may show up. It
%  usually has no significance as long as the value is not more than a
%  few points.
%
% \item[] {\hspace*{-\labelsep}\tt LaTeX Warning: I moved some lines
%  to the next page}
%
%  As mentioned above, \mc{} sometimes screws up the footnote
%  numbering. As a precaution, whenever there is a footnote on a
%  page where \mc{} had to leave a remainder for the following
%  page this warning appears. Check the footnote numbering on this
%  page. If it turns out that it is wrong, you have to manually break
%  the page using |\newpage| or |\pagebreak[..]|.
%
% \item[] {\hspace*{-\labelsep}\tt Floats and marginpars not allowed
%  inside `multicols' environment!}
%
%  This message appears if you try to use the |\marginpar| command or
%  an unstarred version of the {\sf figure} or {\sf table} environment.
%  Such floats will disappear!
%
% \item[] {\hspace*{-\labelsep}\tt Very deep columns!
%          Grid alignment might be broken}
%
% This message can only appear if the option \texttt{grid} was
%    chosen. In that case it will show up if a column has a very large
%    depth so that \mc{} is unable to back up to its baseline. This
%    is only relevant if one tries to produce a document where all
%    text lines are aligned at an invisible grid, something that
%    requires careful adjustment of many parameters and macros, e.g.,
%    heading definitions.
%
% \end{description}
%
%
% \subsection{Tracing the output}
%
% To understand the reasoning behind the decisions \TeX{} makes when
% processing a \mc{} environment, a tracing mechanism is provided.
% If you set the counter `\texttt{tracingmulticols}'
% to a positive \meta{number} you then
% will get some tracing information on the terminal and in the
% transcript file:
% \begin{description}
% \item[$\meta{number}=1$.]  \TeX{} will now tell you, whenever it
%    enters or leaves a \mc{} environment, the number of columns it
%    is working on and its decision about starting a new page before
%    or after the environment.
% \item[$\meta{number}=2$.]
%    In this case you also get information from the balancing routine:
%    the heights tried for the left and right-most columns,
%    information about shrinking if the |\raggedcolumns|
%    declaration is in force and the value of the `{\sf unbalance}'
%    counter if positive.
% \item[$\meta{number}= 3$.]  Setting \meta{number}\pagebreak[2] to
%    this value will additionally trace the mark handling
%    algorithm. It will show what marks are found, what marks are
%    considered, etc. To fully understand this information you will
%    probably have to read carefully trough the implementation.
% \item[$\meta{number}\geq 4$.]  Setting \meta{number}\pagebreak[2] to
%    such a high value will additionally place an |\hrule| into your
%    output, separating the part of text which had already been
%    considered on the previous page from the rest.  Clearly this
%    setting should {\em not\/} be used for the final output.  It will
%    also activate even more debugging code for mark handling.
% \end{description}
%
%
% \end{multicols}
%
% \begin{multicols}{3}[\section{Prefaces to older versions}
%                      \subsection{Preface to version 1.4}]
%   \hbadness=10000
%   Beside fixing some bugs as mentioned in the {\sf multicol.bug} file
%   this new release enhances the \mc{} environment by allowing for
%   balancing in arbitrary contexts. It is now, for example, possible
%   to balance text within a \mc{} or a {\sf minipage} as shown in
%   \ref{tab:newcmds} where a {\sf multicols} environment within a
%   {\sf quote} environment was used. It is now even possible to nest
%   \mc{} environments.
%
%   The only restriction to such inner \mc{} environments (nested, or
%   within \TeX's internal vertical mode) is that such variants will
%   produce a box with the balanced material in it, so that they can
%   not be broken across pages or columns.
%
%   Additionally I rewrote the algorithm for balancing so that it will
%   now produce slightly better results.
%
%   I updated the source documentation but like to apologize in
%   advance for some `left over' parts that slipped through the
%   revision.
%
%   A note to people who like to improve the balancing algorithm of
%   \mc{}: The balancing routine is now placed into a single macro
%   which is called |\balance@columns|. This means that one can easily
%   try different balancing routines by rewriting this macro. The
%   interface for it is explained in table \ref{tab:balance}. There are
%   several improvements possible, one can think of integrating the
%   |\badness| function of \TeX3, define a faster algorithm for finding
%   the right column height, etc. If somebody thinks he/she has an
%   enhancement I would be pleased to learn about it. But please obey
%   the copyright notice and don't change {\sf multicol.dtx} directly!
%   \begin{table*}
%    \begin{quote}
%    \begin{multicols}{2}
%     \raggedcolumns
%     The macro |\balance@columns| that contains the code for balancing
%     gathered material is a macro without parameters. It assumes that
%     the material for balancing is stored in the box |\mult@box| which
%     is a |\vbox|. It also ``knows'' about all parameters set up by the
%     \mc{} environment, like |\col@number|, etc. It can also assume
%     that |\@colroom| is the still available space on the current page.
%
%     When it finishes it must return the individual columns in boxes
%     suitable for further processing with |\page@sofar|. This means
%     that the left column should be stored in box register
%     |\mult@gfirstbox|, the next
%     in register |\mult@firstbox|${}+2$, \ldots,
%     only the last one as an exception in
%     register |\mult@grightbox|. Furthermore it has to set up
%     the two macros
%     |\kept@firstmark| and |\kept@botmark| to hold the values for the
%     first and bottom mark as found in the individual columns. There
%     are some helper functions defined in section \ref{sec:v14} which
%     may be used for this. Getting the marks right ``by hand'' is
%     non-trivial and it may pay off to first take a look at the
%     documentation and implementation of |\balance@columns| below
%     before trying anew.
%    \end{multicols}
%    \end{quote}
%    \vspace*{-3ex}
%    \vspace*{-0ex}
%    \caption{Interface description for \cs{balance@columns}}
%    \label{tab:balance}
%   \end{table*}
% \end{multicols}
%
% \begin{multicols}{3}[\subsection{Preface to version 1.2}]
%   \hbadness=10000
%   After the article about the \mc{} environment was published in
%   \TUB\ 10\#3, I got numerous requests for these macros. However, I
%   also got a changed version of my style file, together with a
%   letter asking me if I would include the changes to get better
%   paragraphing results in the case of narrow lines. The main
%   differences to my original style option were additional parameters
%   (like |\multicoladjdemerits| to be used for |\adjdemerits|, etc.)
%   which would influence the line breaking algorithm.
%
%   But actually resetting such parameters to zero or even worse to a
%   negative value won't give better line breaks inside the \mc{}
%   environment. \TeX{}s line breaking algorithm will only look at
%   those possible line breaks which can be reached without a badness
%   higher than the current value of |\tolerance| (or |\pretolerance|
%   in the first pass). If this isn't possible, then, as a last
%   resort, \TeX{} will produce overfull boxes. All those (and only
%   those) possible break points will be considered and finally the
%   sequence which results in the fewest demerits will be chosen. This
%   means that a value of $-1000$ for |\adjdemerits| instructs \TeX{}
%   to prefer visibly incompatible lines instead of producing better
%   line breaks.
%
%   However, with \TeX{} 3.0 it is possible to get decent line breaks
%   even in small columns by setting |\emergencystretch| to an
%   appropriate value. I implemented a version which is capable of
%   running both in the old and the new \TeX{} (actually it will
%   simply ignore the new feature if it is not available). The
%   calculation of |\emergencystretch| is probably incorrect. I
%   made a few tests but of course one has to have much more experience
%   with the new possibilities to achieve the maximum quality.
%
%   Version 1.1a had a nice `feature': the penalty for using the
%   forbidden floats was their ultimate removal from \LaTeX{}s
%   |\@freelist| so that after a few |\marginpar|s inside the \mc{}
%   environment floats where disabled forever.  (Thanks to Chris
%   Rowley for pointing this out.) I removed this misbehaviour and at
%   the same time decided to allow at least floats spanning all
%   columns, e.g., generated by the |figure*| environment.  You can
%   see the new functionality in table~\ref{tab:newcmds} which was
%   inserted at this very point.
%   \begin{table*}
%    \small
%    \setlength{\multicolsep}{0pt}
%    \begin{quote}
%     \begin{multicols}{2}
%      |\setemergencystretch|: This is a hook for people who like
%      to play around. It is supposed to set the
%      |\emergencystretch| \meta{dimen} register provided in the
%      new \TeX{} 3.0. The first argument is the number of columns and
%      the second one is the current |\hsize|. At the moment the
%      default definition is $4\mbox{\tt pt} \times |#1|$, i.e.\ the
%      |\hsize| isn't used at all. But maybe there are better
%      formulae.
%
%      \setlength{\emergencystretch}{20pt} |\set@floatcmds|: This is
%      the hook for the experts who like to implement a full float
%      mechanism for the \mc{} environment. The |@| in the name
%      should signal that this might not be easy.
%    \end{multicols}
%   \end{quote}
%   \vspace*{-1ex}
%   \vspace*{-0ex}
%   \caption[]{The new commands of {\sf multicol.sty} version 1.2.
%            Both commands might be removed if good solutions to these
%            open problems are found. I hope that these commands will
%            prevent that nearly identical style files derived from
%            this one are floating around.}
%   \label{tab:newcmds}
%   \end{table*}
%   However single column floats are still forbidden and I don't think
%   I will have time to tackle this problem in the near future. As an
%   advice for all who want to try: wait for \TeX{} 3.0. It has a few
%   features which will make life much easier in multi-column
%   surroundings. Nevertheless we are working here at the edge of
%   \TeX{}s capabilities, really perfect solutions would need a
%   different approach than it was done in \TeX{}s page builder.
%
%   The text below is nearly unchanged, I only added documentation at
%   places where new code was added.
% \end{multicols}
%
% \changes{v1.5l}{1996/01/13}{Try hard to explain unresolved reference
%                that happens if \cs{OnlyDescription} is used}
%
% \StopEventually{\PrintIndex \PrintChanges
%   \ifx\Finale\relax
%    \typeout{**********************************}
%    \typeout{* Info: Typesetting this document with
%                     \protect\OnlyDescription\space will}
%    \typeout{* Info: result in one unresolved
%                     reference to `sec:v14'.}
%    \typeout{* Info: --- tough, it's just not there in this case!}
%    \typeout{**********************************}
%   \fi
%  }
%
%
% \begin{multicols}{2}[\section{The Implementation}
%         We are now switching to two-column output to show the
%         abilities of this environment (and bad layout decisions).
%      \subsection{The documentation driver file}
%         ][10\baselineskip]
%
%   \hbadness=10000
%
%
% The next bit of code contains the documentation driver file for
% \TeX{}, i.e., the file that will produce the documentation you are
% currently reading. It will be extracted from this file by the {\tt
% docstrip} program.
% Since this is the first code in this file one can produce the
% documentation
% simply by running \LaTeX{} on the \texttt{.dtx} file.
%    \begin{macrocode}
%<*driver>
\documentclass{ltxdoc}
%    \end{macrocode}
%    We use the \texttt{balancingshow} option when loading \mc{} so
%    that full tracing is produced. This has to be done before the
%    \texttt{doc} package is loaded, since \texttt{doc} otherwise
%    requires \mc{} without any options.
%    \begin{macrocode}
\usepackage{multicol}[1999/05/25]
\usepackage{doc}
%    \end{macrocode}
%    First we set up the page layout suitable for
%    this article.
%    \begin{macrocode}
\setlength{\textwidth}{39pc}
\setlength{\textheight}{54pc}
\setlength{\parindent}{1em}
\setlength{\parskip}{0pt plus 1pt}
\setlength{\oddsidemargin}{0pc}
\setlength{\marginparwidth}{0pc}
\setlength{\topmargin}{-2.5pc}
\setlength{\headsep}{20pt}
\setlength{\columnsep}{1.5pc}
%    \end{macrocode}
%    We want a rule between columns.
%    \begin{macrocode}
\setlength\columnseprule{.4pt}
%    \end{macrocode}
%    We also want to ensure that a new \mc{} environment finds enough
%    space at the bottom of the page.
%    \begin{macrocode}
\setlength\premulticols{6\baselineskip}
%    \end{macrocode}
%    When balancing columns we disregard solutions that
%    are too bad. Also, if the last column is too bad
%    we typeset it without stretch.
%    \begin{macrocode}
\setcounter{columnbadness}{7000}
\setcounter{finalcolumnbadness}{7000}
%    \end{macrocode}
%    The index is supposed to come out in four columns.
%    And we don't show macro names in the margin.
%    \begin{macrocode}
\setcounter{IndexColumns}{4}
\let\DescribeMacro\SpecialUsageIndex
\let\DescribeEnv\SpecialEnvIndex
\renewcommand\PrintMacroName[1]{}
\CodelineIndex
%\DisableCrossrefs           % Partial index
\RecordChanges               % Change log
%    \end{macrocode}
%    Line numbers are very small for this article.
%    \begin{macrocode}
\renewcommand{\theCodelineNo}
  {\scriptsize\rm\arabic{CodelineNo}}
\settowidth\MacroIndent{\scriptsize\rm 00\ }

\begin{document}
  \typeout
   {****************************************
 ^^J* Expect some Under- and overfull boxes.
 ^^J****************************************}
   \DocInput{multicol.dtx}
\end{document}
%</driver>
%    \end{macrocode}
%
%
% \end{multicols}
%
% \begin{multicols}{2}[\subsection{Identification and
%                      option processing}]
%
%
% We start by identifying the package. Since it makes use of features
% only available in \LaTeXe{} we ensure that this format is available.
% (Now this is done earlier in the file.)
%    \begin{macrocode}
%<*package>
% \NeedsTeXFormat{LaTeX2e}
% \ProvidesPackage{multicol}[..../../..
%    v... multicolumn formatting]
%    \end{macrocode}
%
%^^A \subsection{Option processing}
%
%    Next we declare options supported by \mc{}. Two-column mode
%    and \mc{} do not work together so we warn about possible
%    problems. However, since you can revert to |\onecolumn|
%    in which case \mc{} does work, we don't make this an error.
%    \begin{macrocode}
\DeclareOption{twocolumn}
   {\PackageWarning{multicol}{May not work
             with the twocolumn option}}
%    \end{macrocode}
%    Tracing is done using a counter. However
%    it is also possible to invoke the tracing
%    using the options declared below.
%    \begin{macrocode}
\newcount\c@tracingmulticols
\DeclareOption{errorshow}
    {\c@tracingmulticols\z@}
\DeclareOption{infoshow}
    {\c@tracingmulticols\@ne}
\DeclareOption{balancingshow}
    {\c@tracingmulticols\tw@}
\DeclareOption{markshow}
    {\c@tracingmulticols\thr@@}
\DeclareOption{debugshow}
    {\c@tracingmulticols5\relax}
%    \end{macrocode}
%    The next option is intended for typesetting on a |\baselineskip|
%    grid. Right now it doesn't do anything other than warning if it
%    thinks that the grid got lost.
% \changes{v1.6a}{2003/03/15}{New option grid}
%    \begin{macrocode}
\let\mc@gridwarn\maxdimen
\DeclareOption{grid}{\def\mc@gridwarn{\maxdepth}}
%    \end{macrocode}
%
% \changes{v1.8e}{2014/04/23}{Support \cs{docolaction}}
%    Next option enables the |\docolaction| command. As this changes
%   the \texttt{.aux} file content this is not automatically enabled.
%    \begin{macrocode}
\DeclareOption{colaction}{%
  \def\mc@col@status@write{%
         \protected@write\@auxout{}%
              {\string\mc@col@status
              {\ifmc@firstcol 1\else 2\fi}}%
         \mc@firstcolfalse}%
  \def\mc@lastcol@status@write{%
         \protected@write\@auxout{}%
              {\string\mc@col@status{3}}}%
}
\let\mc@col@status@write\relax
\let\mc@lastcol@status@write\relax
%    \end{macrocode}
%
%    \begin{macrocode}
\ProcessOptions
%    \end{macrocode}
%
%
% \end{multicols}
%
% \begin{multicols}{2}[\subsection{Starting and
%                      Ending the \mc{} Environment}]
%
% \begin{macro}{\multicols}
%    As mentioned before, the \mc{} environment has one mandatory
%    argument (the number of columns) and up to two optional ones.  We
%    start by reading the number of columns into the |\col@number|
%    register.
%    \begin{macrocode}
\def\multicols#1{\col@number#1\relax
%    \end{macrocode}
%    If the user forgot the argument, \TeX{} will complain about a
%    missing number at this point. The error recovery mechanism will
%    then use zero, which isn't a good choice in this case. So we
%    should now test whether everything is okay. The minimum is two
%    columns at the moment.
% \changes{v1.3b}{1990/10/09}{Minimum of two columns}
%    \begin{macrocode}
  \ifnum\col@number<\tw@
     \PackageWarning{multicol}%
      {Using `\number\col@number'
       columns doesn't seem a good idea.^^J
       I therefore use two columns instead}%
     \col@number\tw@ \fi
%    \end{macrocode}
%    We have only enough box registers for ten columns, so we need to
%    check that the user hasn't asked for more.
% \changes{v1.4k}{1992/06/27}{Maximum of 5 columns (temp)}
% \changes{v1.5a}{1992/11/04}{Allow 10 columns again}
%    \begin{macrocode}
  \ifnum\col@number>10
     \PackageError{multicol}%
      {Too many columns}%
      {Current implementation doesn't
       support more than 10 columns.%
       \MessageBreak
       I therefore use 10 columns instead}%
     \col@number10 \fi
%    \end{macrocode}
%    Within the environment we need a special version of the
%    kernel |\@footnotetext| command since the original sets the
%    the |\hsize| to |\columnwidth| which is not correct in the
%    multicol environment. Here |\columnwidth| refers to the width of
%    the individual column and the footnote should be in |\textwidth|.
%    Since |\@footnotetext| has a different definition inside a
%    minipage environment we do not redefine it directly. Instead we
%    locally set |\columnwidth| to |\textwidth| and call the original
%    (current) definition stored in |\orig@footnotetext|.
% \changes{v1.5p}{1997/12/14}{Redefinition of \cs{@footnotetext}
%                           only within env pr/2689.}
% \changes{v1.5r}{1998/08/17}{Use \cs{@footnotetext} but with
%                             local change to \cs{columnwidth}.}
% \changes{v1.5w}{1999/10/21}{Make \cs{@footnotetext} long to allow
%                             multi-paragraph footnotes.}
%    If the \mc{} environment is nested inside another \mc{}
%    environment then the redefinition has already happened. So be
%    better test for this situation. Otherwise, we will get a \TeX{}
%    stack overflow as this would generate a self-referencing definition.
% \changes{v1.6e}{2004/02/14}{Avoid self-referencing definition of
%    \cs{@footnotetext} (pr/3618)}.
%    \begin{macrocode}
     \ifx\@footnotetext\mult@footnotetext\else
       \let\orig@footnotetext\@footnotetext
       \let\@footnotetext\mult@footnotetext
     \fi
%    \end{macrocode}
%    Now we can safely look for the optional arguments.
%    \begin{macrocode}
  \@ifnextchar[\mult@cols{\mult@cols[]}}
%    \end{macrocode}
% \end{macro}
%
%
%
% \begin{macro}{\mult@footnotetext}
%
%    \begin{macrocode}
\long\def\mult@footnotetext#1{\begingroup
         \columnwidth\textwidth
         \orig@footnotetext{#1}\endgroup}
%    \end{macrocode}
% \end{macro}
%
% \begin{macro}{\mult@cols}
%    The |\mult@cols| macro grabs the first optional argument
%    (if any) and looks for the second one.
%    \begin{macrocode}
\def\mult@cols[#1]{\@ifnextchar[%
%    \end{macrocode}
%    This argument should be a \meta{dimen} denoting the minimum free
%    space needed on the current page to start the environment. If the
%    user didn't supply one, we use |\premulticols| as a
%    default.
%    \begin{macrocode}
  {\mult@@cols{#1}}%
  {\mult@@cols{#1}[\premulticols]}}
%    \end{macrocode}
% \end{macro}
%
%
% \begin{macro}{\mult@@cols}
%    After removing all arguments from the input we are able
%    to start with |\mult@@cols|.
%    \begin{macrocode}
\def\mult@@cols#1[#2]{%
%    \end{macrocode}
%    First thing we do is to decide whether or not this is an
%    unbounded multicols environment, i.e. one that may split across
%    pages, or one that has to be typeset into a box. If we are in
%    \TeX's ``inner'' mode (e.g., inside a box already) then we have a
%    boxed version of multicols therefore we set the |@boxedmulticols|
%    switch to true.  The \mc{} should start in vertical mode. If we
%    are not already there we now force it with |\par| since otherwise
%    the test for ``inner'' mode wouldn't show if we are in a box.
% \changes{v1.4f}{1992/04/28}{\cs{par} added to allow for correct inner test}
%    \begin{macrocode}
  \par
  \ifinner \@boxedmulticolstrue
%    \end{macrocode}
%    Otherwise we check |\doublecol@number|. This counter is zero
%    outside a multicols environment but positive inside (this happens
%    a little later on). In the second case we need to process the
%    current multicols also in ``boxed mode'' and so change the switch
%    accordingly.
%    \begin{macrocode}
  \else
    \ifnum \doublecol@number>\z@
       \@boxedmulticolstrue
    \fi
  \fi
%    \end{macrocode}
%    Then we look to see if statistics are requested:
%    \begin{macrocode}
  \mult@info\z@
      {Starting environment with
       \the\col@number\space columns%
%    \end{macrocode}
%    In boxed mode we add some more info.
% \changes{v1.4f}{1992/04/28}{\cs{on@line} added to tracing info}
%    \begin{macrocode}
        \if@boxedmulticols\MessageBreak
           (boxed mode)\fi
      }%
%    \end{macrocode}
%    Then we measure the current page to see whether a useful portion
%    of the multicolumn environment can be typeset.  This routine
%    might start a new page.
% \changes{v1.4a}{1992/02/14}{Forgotten braces added}
%    \begin{macrocode}
   \enough@room{#2}%
%    \end{macrocode}
%    Now we output the first argument and produce vertical space
%    above the columns. (Note that this argument corresponds to the
%    first optional argument of the {\sf multicols} environment.)
%    For many releases this argument was typeset in a group to get
%    a similar effect as |\twocolumn[..]| where the argument is
%    also implicitly surrounded by braces. However, this conflicts
%    with local changes done by things like sectioning commands (which
%    account for the majority of commands used in that argument)
%    messing up vertical spacing etc.\ later in the document so that
%    from version v1.5q on this argument is again typeset at the outer
%    level.
% \changes{v1.4e}{1992/03/16}{Typeset optional arg inside group}
% \changes{v1.5q}{1998/01/19}{And removed the group again six years later}
%    \begin{macrocode}
   #1\par\addvspace\multicolsep
%    \end{macrocode}
%    When the last line of a paragraph had a positive depth then this
%    depth normally taken into account by the baselineskip calculation
%    for the next line. However, the columns produced by a following
%    \texttt{multicol} are rigid and thus the distance from the
%    baseline of a previous text line to the first line in a
%    \texttt{multicol} would differ depending on the depth of the
%    previous line. To account for this we add a negative space unless
%    the depth is \texttt{-1000pt} which signals something special to
%    \TeX and is not supposed to be a real depth.
% \changes{v1.5z1}{2003/02/17}{Add a kern to cancel potential depth of
%    previous line}
%    \begin{macrocode}
   \ifdim \prevdepth = -\@m\p@
   \else
%    \end{macrocode}
%    The actual generation of this corrective space is a little bit
%    more complicated as it doesn't make sense to always back up to
%    the previous baseline (in case an object with a very large depth
%    was placed there, e.g., a centered tabular). So we only back up
%    to the extend that we are within the |\baselineskip| grid. We
%    know that the box produced by \mc{} has |\topskip| at its top so
%    that also needs to be taken into account.
% \changes{v1.6a}{2003/03/15}{Adjust spacing}
%    \begin{macrocode}
     \@tempcnta\prevdepth
     \@tempcntb\baselineskip
     \divide\@tempcnta\@tempcntb
     \advance\@tempcnta\@ne
     \dimen@\prevdepth
     \advance\dimen@ -\@tempcnta\baselineskip
     \advance\dimen@ \topskip
     \kern-\dimen@
   \fi
%    \end{macrocode}
%    We start a new grouping level to hide all subsequent changes
%    (done in |\prepare@multicols| for example).
%    \begin{macrocode}
   \begingroup
     \prepare@multicols
%    \end{macrocode}
%    If we are in boxed mode we now open a box to typeset all material
%    from the multicols body into it, otherwise we simply go ahead.
% \changes{v1.4g}{1992/05/07}{\cs{global} was probably wrong but at least
%                           unnecessary}
%    \begin{macrocode}
     \if@boxedmulticols
       \setbox\mult@box\vbox\bgroup
%    \end{macrocode}
%
% \changes{v1.8c}{2013/11/03}{Add \cs{color@setgroup} to prevent color leak}
%    \begin{macrocode}
                            \color@setgroup
%    \end{macrocode}
% \changes{v1.5?}{1994/?/?}{Penalty moved to later point}
%    We may have to reset some parameters at this point,
%    perhaps |\@parboxrestore|
%    would be the right action but I leave it for the moment.
% \changes{v1.4l}{1992/08/17}{\cs{@totalleftmargin} now in
%                             \cs{prepare@multicols}}
%    \begin{macrocode}
     \fi
%    \end{macrocode}
%    We finish by suppressing initial spaces.
%    \begin{macrocode}
     \ignorespaces}
%    \end{macrocode}
% \end{macro}
%
%  \begin{macro}{\if@boxedmulticols}
%    Here is the switch and the box for ``boxed'' multicols code.
%    \begin{macrocode}
\newif\if@boxedmulticols
\@boxedmulticolsfalse
\newbox\mult@box
%    \end{macrocode}
%  \end{macro}
%
% \begin{macro}{\enough@room}
% \changes{v1.0c}{1989/05/12}{Penalty 0 added to empty the contribution
%                           list.}
%    The |\enough@room| macro used
%    above isn't perfect but works reasonably well in this context. We
%    measure the free space on the current page by subtracting
%    |\pagetotal| from |\pagegoal|. This isn't entirely
%    correct since it doesn't take the `shrinking' (i.e.\
%    |\pageshrink|) into account.  The `recent contribution list'
%    might be nonempty so we start with |\par| and an explicit
%    |\penalty|.\footnote{See the documentation of
%    \cs{endmulticols} for further details.}
%    Actually, we use |\addpenalty| to ensure that a following
%    |\addvspace| will `see' the vertical space that might be
%    present.
%    The use of |\addpenalty| will have the effect that all items from
%    the recent contributions will be moved to the main vertical list
%    and the |\pagetotal| value will be updated correctly. However,
%    the penalty will be placed in front of any dangling glue item
%    with the result that the main vertical list may already be
%    overfull even if \TeX{} is not invoking the output routine.
% \changes{v1.3b}{1990/10/09}{Do \cs{penalty} with \cs{addpenalty}}
% \changes{v1.4e}{1992/03/16}{But ignore \cs{@nobreak} in \cs{addpenalty}}
%    \begin{macrocode}
\def\enough@room#1{%
%    \end{macrocode}
%    Measuring makes only sense when we are not in ``boxed mode'' so
%    the routine does nothing if the switch is true.
%    \begin{macrocode}
   \if@boxedmulticols\else
   \par
%    \end{macrocode}
%    \label{mac:enoughroom}
%    To empty the contribution list the first release contained a
%    penalty zero but this had the result that |\addvspace| couldn't
%    detect preceding glue. So this was changed to |\addpenalty|. But
%    this turned out to be not enough as |\addpenalty| will not add a
%    penalty when |@nobreak| is true. Therefore we force this switch
%    locally to false. As a result there may be a break between
%    preceding text and the start of a multicols environment, but this
%    seems acceptable since there is the optional argument for exactly
%    this reason.
%    \begin{macrocode}
   \bgroup\@nobreakfalse\addpenalty\z@\egroup
   \page@free \pagegoal
   \advance \page@free -\pagetotal
%    \end{macrocode}
%    To be able to output the value we need to assign it to a register
%    first since it might be a register (default) in which case we
%    need to use |\the| or it might be a plain value in which case
%    |\the| would be wrong.
% \changes{v1.5e}{1994/05/26}{Assign arg to skip register to be able
%                             to output value}
%    \begin{macrocode}
     \@tempskipa#1\relax
%    \end{macrocode}
%    Now we test whether tracing information is required:
%    \begin{macrocode}
   \mult@info\z@
       {Current page:\MessageBreak
        height=%
        \the\pagegoal: used \the\pagetotal
        \space -> free=\the\page@free
        \MessageBreak
        needed \the\@tempskipa
              \space(for #1)}%
%    \end{macrocode}
%    Our last action is to force a page break if there isn't enough
%    room left.
%    \begin{macrocode}
   \ifdim \page@free <#1\newpage \fi
  \fi}
%    \end{macrocode}
% \end{macro}
%
% \begin{macro}{\prepare@multicols}
%    When preparing for multicolumn output several things must
%    be done.
%    \begin{macrocode}
\def\prepare@multicols{%
%    \end{macrocode}
%    We start saving the current |\@totalleftmargin| and then
%    resetting the |\parshape| in case we are inside some list
%    environment. The correct indentation for the \mc{} environment in
%    such a case will be produced by moving the result to the right by
%    |\multicol@leftmargin| later on. If we would use the value of of
%    |\@totalleftmargin| directly then lists inside the \mc{}
%    environment could cause a shift of the output.
% \changes{v1.4l}{1992/08/17}{saved \cs{@totalleftmargin}}
%    \begin{macrocode}
  \multicol@leftmargin\@totalleftmargin
  \@totalleftmargin\z@
  \parshape\z@
%    \end{macrocode}
%    We also set the register |\doublecol@number| for later use.  This
%    register should contain $2\times |\col@number|$.  This is also an
%    indicator that we are within a \mc{} environment as mentioned
%    above.
% \changes{v1.5a}{1992/11/04}{Add offset to \cs{doublecolnumber}}
%    \begin{macrocode}
  \doublecol@number\col@number
  \multiply\doublecol@number\tw@
  \advance\doublecol@number\mult@rightbox
%    \end{macrocode}
%
%    \begin{macrocode}
  \if@boxedmulticols
    \let\l@kept@firstmark\kept@firstmark
    \let\l@kept@botmark\kept@botmark
    \global\let\kept@firstmark\@empty
    \global\let\kept@botmark\@empty
  \else
%    \end{macrocode}
%    We add an empty box to the main vertical list to ensure that we
%    catch any insertions (held over or inserted at the top of the
%    page). Otherwise it might happen that the |\eject| is discarded
%    without calling the output routine. Inside the output routine we
%    remove this box again.  Again this code applies only if we are on
%    the main vertical list and not within a box.
%    However, it is not enough to turn off interline spacing, we also
%    have to clear |\topskip| before adding this box, since |\topskip|
%    is always inserted before the first box on a page which would
%    leave us with an extra space of |\topskip| if \mc{} start on a
%    fresh sheet.
% \changes{v1.3c}{1991/03/03}{\cs{null} inserted and removed in output}
% \changes{v1.4a}{1992/02/11}{Conditional code for boxed mode added.}
% \changes{v1.4o}{1992/11/22}{\cs{topskip} locally zeroed.}
%    \begin{macrocode}
    \nointerlineskip {\topskip\z@\null}%
    \output{%
      \global\setbox\partial@page\vbox
        {%
%    \end{macrocode}
%    Now we have to make sure that we catch one special situation which
%    may result in loss of text! If the user has a huge amount of
%    vertical material within the first optional argument that is larger
%    then |\premulticols| and we are near the bottom of the page then it
%    can happen that not the |\eject| is triggering this special output
%    routine but rather the overfull main vertical list.  In that case
%    we get another breakpoint through the |\eject| penalty. As a result
%    this special output routine would be called twice and the contents
%    of |\partial@page|, i.e.\ the material before the \mc{}
%    environment gets lost. There are several solutions to avoid this
%    problem, but for now we will simply detect this and inform the user
%    that he/she has to enlarge the |\premulticols| by using a suitable
%    value for the second argument.
% \changes{v1.4a}{1992/02/11}{Checking for text losses.}
%    \begin{macrocode}
%<*check>
         \ifvoid\partial@page\else
           \PackageError{multicol}%
            {Error saving partial page}%
            {The part of the page before
             the multicols environment was
             nearly full with^^Jthe result
             that starting the environment
             will produce an overfull
             page.  Some^^Jtext may be lost!
             Please increase \premulticols
             either generally or for this%
             ^^Jenvironment by specifying a
             suitable value in the second
             optional argument to^^Jthe
             multicols environment.}
           \unvbox\partial@page
           \box\last@line
         \fi
%</check>
         \unvbox\@cclv
         \global\setbox\last@line\lastbox
        }%
%    \end{macrocode}
%    Finally we need to record the marks that are present within the
%    |\partial@page| so that we can construct correct first and bottom
%    marks later on. This is done by the following code.
% \changes{v1.4a}{1992/02/14}{kept marks initiated}
%    \begin{macrocode}
           \prep@keptmarks
%    \end{macrocode}
%    Finally we have to initialize |\kept@topmark| which should
%    ideally be initialized with the mark that is current on ``top''
%    of this page. Unfortunately we can't use |\topmark| because this
%    register will not always contain what its name promises because
%    \LaTeX{} sometimes calls the output routine for float
%    management.\footnote{During such a call the \cs{botmark}
%    gets globally copied to \cs{topmark} by the \TeX{}
%    program.} Therefore we use the second best solution by
%    initializing it with |\firstmark|. In fact, for our purpose this
%    doesn't matter as we use |\kept@topmark| only to initialize
%    |\firstmark| and |\botmark| of a following page if we don't find
%    any marks on the current one.
% \changes{v1.4i}{1992/06/18}{\cs{kept@topmark} initialized.}
%    \begin{macrocode}
           \global\let\kept@topmark\firstmark
          }\eject
%    \end{macrocode}
%    The next thing to do is to assign a new value to |\vsize|.
%    \LaTeX{} maintains the free room on the page (i.e.\ the page
%    height without the space for already contributed floats) in the
%    register |\@colroom|. We must subtract the height of
%    |\partial@page| to put the actual free room into this
%    variable.
%    \begin{macrocode}
    \advance\@colroom-\ht\partial@page
%    \end{macrocode}
%    Then we have to calculate the |\vsize| value to use during column
%    assembly. |\set@mult@vsize| takes an argument which allows to
%    make the setting local (|\relax|) or global (|\global|). The
%    latter variant is used inside the output routine below. At this
%    point here we have to make a local change to |\vsize| because we
%    want to get the original value for |\vsize| restored in case
%    this \mc{} environment ends on the same page where it has started.
% \changes{v1.4p}{1992/11/26}{Use different \cs{vsize} setting}
% \changes{v1.4p}{1992/11/26}{Code moved to \cs{set@mult@vsize}}
%    \begin{macrocode}
    \set@mult@vsize\relax
%    \end{macrocode}
%    Now we switch to a new |\output| routine which will be used
%    to put the gathered column material together.
%    \begin{macrocode}
    \output{\multi@column@out}%
%    \end{macrocode}
%    Finally we handle the footnote insertions. We have to multiply
%    the magnification factor and the extra skip by the number of
%    columns since each footnote reduces the space for every column
%    (remember that we have pagewide footnotes).  If, on the other
%    hand, footnotes are typeset at the very end of the document, our
%    scheme still works since |\count\footins| is zero then, so it
%    will not change.
%    To allow even further customization the setting of the |\footins|
%    parameters is done in a separate macro.
% \changes{v1.5?}{1994/?/?}{Use \cs{init@mult@footins}}
%    \begin{macrocode}
    \init@mult@footins
%    \end{macrocode}
%    For the same reason (pagewide footnotes), the \meta{dimen}
%    register controlling the maximum space used for footnotes isn't
%    changed.  Having done this, we must reinsert all the footnotes
%    which are already present (i.e.\ those encountered when the
%    material saved in |\partial@page| was first processed). This
%    will reduce the free space (i.e.\ |\pagetotal|) by the
%    appropriate amount since we have changed the magnification
%    factor, etc.\ above.
%    \begin{macrocode}
    \reinsert@footnotes
%    \end{macrocode}
%    Inside \mc{} a |\clearpage| is fairly useless as we aren't supporting
%    floats. In fact, it can cause harm as it doesn't know about the
%    |\partial@page| and may therefore result in making columns too long.
%    So we change that to bahve like |\newpage| but also check if there
%    are any deferred floats. If so, perhaps the user tried to place them through
%    that |\clearpage| (but that needs to be done before starting the \mc{}
%    environment.
% \changes{v1.8q}{2017/04/11}{Make \cs{clearpage} behave like \cs{newpage} (pr/4511)}
%    \begin{macrocode}
   \def\clearpage{%
     \ifx\@deferlist\@empty\else
       \PackageError{multicol}%
          {Deferred floats not cleared}%
          {A \string\clearpage\space inside multicols acts like
           \string\newpage\space and doesn't clear floats.\MessageBreak
           Move it before the multicols environment if you need it.}%
     \fi
    \newpage}%
%    \end{macrocode}
%    All the code above was only necessary for the unrestricted \mc{}
%    version, i.e.\ the one that allows page breaks. If we are within
%    a box there is no point in setting up special output routines or
%    |\vsize|, etc.
%    \begin{macrocode}
  \fi
%    \end{macrocode}
%    But now we are coming to code that is necessary in all cases.  We
%    assign new values to |\vbadness|, |\hbadness| and |\tolerance|
%    since it's rather hard for \TeX{} to produce `good' paragraphs
%    within narrow columns.
% \changes{v1.2a}{1990/02/05}{\cs{vbadness} 10001 now.}
%    \begin{macrocode}
  \vbadness\@Mi \hbadness5000
  \tolerance\multicoltolerance
%    \end{macrocode}
%    Since nearly always the first pass will fail we ignore it
%    completely telling \TeX{} to hyphenate directly.  In fact, we now
%    use another register to keep the value for the multicol
%    pre-tolerance, so that a designer may allow to use
%    |\pretolerance|.
% \changes{v1.2a}{1990/02/05}{\cs{pretolerance} -1 because it nearly never
%                            succeeds.}
% \changes{v1.4e}{1992/03/20}{Using}
%    \begin{macrocode}
  \pretolerance\multicolpretolerance
%    \end{macrocode}
%    For use with the new \TeX{} we set |\emergencystretch| to
%    $|\col@number| \times 4pt$. However this is only a guess
%    so at the moment this is done in a macro
%    |\setemergencystretch| which gets the current |\hsize|
%    and the number of columns as arguments. Therefore users are able
%    to figure out their own formula.
% \changes{v1.2a}{1990/02/05}{\cs{setemergencystretch} added.}
%    \begin{macrocode}
  \setemergencystretch\col@number\hsize
%    \end{macrocode}
%    Another hook to allow people adding their own extensions without
%    making a new package is |\set@floatcmds| which handles any
%    redefinitions of \LaTeX{}s internal float commands to work with
%    the \mc{} environment. At the moment it is only used to redefine
%    |\@dblfloat| and |\end@dblfloat|.
% \changes{v1.2a}{1990/02/05}{\cs{set@floatcmds} added.}
%    \begin{macrocode}
  \set@floatcmds
%    \end{macrocode}
%    Additionally, we advance |\baselineskip| by
%    |\multicolbaselineskip| to allow corrections for narrow
%    columns.
%    \begin{macrocode}
  \advance\baselineskip\multicolbaselineskip
%    \end{macrocode}
% \changes{v1.0e}{1989/06/21}{\cs{textwidth} changed to \cs{linewidth}.}
% \changes{v1.0e}{1989/06/21}{So this file will
%                           work with the `twocolumn' command.}
%    The |\hsize| of the columns is given by the formula:
%    \[
%       { |\linewidth| - (|\col@number|-1)
%                             \times
%                             |\columnsep|
%        \over
%        |\col@number|}
%    \]
%    The formula above has changed from release to release. We now
%    start with the current value of |\linewidth| so that the column
%    width is properly calculated when we are inside a minipage or a
%    list or some other environment.
%    This will be achieved with:
%    \begin{macrocode}
  \hsize\linewidth \advance\hsize\columnsep
  \advance\hsize-\col@number\columnsep
  \divide\hsize\col@number
%    \end{macrocode}
%    We also set |\linewidth| and |\columnwidth| to |\hsize| In the past
%    |\columnwidth| was left unchanged. This is inconsistent,
%    but |\columnwidth| is used only by floats (which
%    aren't allowed in their current implementation) and by the
%    |\footnote| macro. Since we want pagewide
% footnotes\footnote{I'm not sure that I really want pagewide
%                       footnotes. But balancing of the last page can
%                       only be achieved with this approach or with a
%                       multi-path algorithm which is complicated and
%                       slow. But it's a challenge to everybody to
%                       prove me wrong!  Another possibility is to
%                       reimplement a small part of the {\it
%                       fire\_up\/} procedure in \TeX{} (the program).
%                       I think that this is the best solution if you
%                       are interested in complex page makeup, but it
%                       has the disadvantage that the resulting
%                       program cannot be called \TeX{} thereafter.}
%    this simple trick saved us from rewriting the |\footnote|
%    macros. However, some applications referred to |\columnwidth| as
%    the ``width of the current column'' to typeset displays
%    (the \texttt{amsmath} package, for example) and to allow the use
%    of such applications together with \texttt{multicol} this is now
%    changed.
%
%    Before we change |\linewidth| to the new value we record its old
%    value in some register called |\full@width|. This value is
%    used later on when we package all columns together.
% \changes{v1.0e}{1989/06/21}{Setting of \cs{columnwidth} removed.}
% \changes{v1.5o}{1997/11/16}{Setting of \cs{columnwidth} added again
%                           pr/2664.}
%    \begin{macrocode}
  \full@width\linewidth
  \linewidth\hsize
  \columnwidth\hsize
}
%    \end{macrocode}
% \end{macro}
%
%  \begin{macro}{\init@mult@footins}
%    This macro is used to set up the parameters associated
%    with footnote floats. It can be redefined by applications that
%    require different amount of spaces when typesetting footnotes.
%    \begin{macrocode}
\def\init@mult@footins{%
    \multiply\count\footins\col@number
    \multiply\skip \footins\col@number
}
%    \end{macrocode}
% \end{macro}
%
%  \begin{macro}{\set@mult@vsize}
%
%    Since we have to set |\col@umber| columns on one page,
%    each with a height of |\@colroom|, we have to assign
%    $
%      |\vsize|  = |\col@number| \times |\@colroom|
%    $
%    in order to collect enough material before entering the
%    |\output| routine again. In fact we have to add another
%    $
%       (|\col@number|-1) \times (|\baselineskip|-|\topskip|)
%    $
%    if you think about it.
% \changes{v1.4p}{1992/11/26}{Macro added.}
%    \begin{macrocode}
\def\set@mult@vsize#1{%
    \vsize\@colroom
    \@tempdima\baselineskip
    \advance\@tempdima-\topskip
    \advance\vsize\@tempdima
    \vsize\col@number\vsize
    \advance\vsize-\@tempdima
%    \end{macrocode}
%    But this might not be enough since we use |\vsplit| later to
%    extract the columns from the gathered material. Therefore we add
%    some `extra lines,' one for each column plus a corrective action
%    depending on the value of the `\texttt{collectmore}' counter.
%    The final value is assigned globally if |#1| is |\global| because
%    we want to use this macro later inside the output routine too.
% \changes{v1.6c}{2003/04/08}{Collect one line per column more}
%    \begin{macrocode}
    \advance\vsize\col@number\baselineskip
    #1\advance\vsize
        \c@collectmore\baselineskip}
%    \end{macrocode}
%  \end{macro}
%
%
%  \begin{macro}{\multicol@leftmargin}
%    Here is the dimen register we need for saving away the outer
%    value of |\@totalleftmargin|.
%    \begin{macrocode}
\newdimen\multicol@leftmargin
%    \end{macrocode}
%  \end{macro}
%
%
%  \begin{macro}{\@Mvi}
%    In versions prior to 1.8r the balancing at the end of the
%   environment was done by changing the output routine from
%   |\multi@column@out| to |\balance@column@out|. As it turned out
%   that this has a couple of issues when the last columns should not
%   be balanced after all (for example because they contained several
%   |\columnbreak| commands we now stay with one output routine for
%   the environment and only signal that we reached the end of the
%   environment by marking it with a special penalty that we can check
%   for later.
% \changes{v1.8r}{2018/01/28}{Macro added}
%    \begin{macrocode}
\mathchardef\@Mvi=10006  % 10005 is \columnbreak
%    \end{macrocode}
%  \end{macro}
%
%
% \begin{macro}{\endmulticols}
%    When the end of the \mc{} environment is sensed we have to
%    balance the gathered material. Depending on whether or not we are
%    inside a boxed multicol different things must happen. But first
%    we end the current paragraph with a |\par| command.
%    \begin{macrocode}
\def\endmulticols{\par
  \if@boxedmulticols
%    \end{macrocode}
%    In boxed mode we have to close the box in which we have gathered
%    all material for the columns. But before we do this we need to
%    remove any space at the end of the box as we don't want to use
%    this in balancing. Because of the |\color@endgroup| this can't be
%    done later in |\balance@columns| as the color command will hide
%    it. 
% \changes{v1.8c}{2013/11/03}{Add \cs{color@endgroup} to prevent color leak}
% \changes{v1.8f}{2014/06/19}{Discard spaces before adding \cs{color@endgroup}}
%    \begin{macrocode}
    \remove@discardable@items\color@endgroup\egroup
%    \end{macrocode}
%    Now we call |\balance@columns| the routine that balances material
%    stored in the box |\mult@box|.
% \changes{v1.5?}{1994/?/?}{Splitting off zero box moved to
%                           \cs{balance@columns}}
%    \begin{macrocode}
    \balance@columns
%    \end{macrocode}
%    After balancing the result has to be returned by the command
%    |\page@sofar|. But before we do this we reinsert any marks found
%    in box |\mult@box|.
%    \begin{macrocode}
    \return@nonemptymark{first}%
                 \kept@firstmark
    \return@nonemptymark{bot}%
                  \kept@botmark
    \page@sofar
%    \end{macrocode}
%
%    \begin{macrocode}
    \global\let\kept@firstmark
               \l@kept@firstmark
    \global\let\kept@botmark
            \l@kept@botmark
%<*marktrace>
    \mult@info\tw@
      {Restore kept marks to\MessageBreak
       first: \meaning\kept@firstmark
       \MessageBreak bot\space\space:
                     \meaning\kept@botmark }%
%</marktrace>
%    \end{macrocode}
%    This finishes the code for the ``boxed'' case.
%    \begin{macrocode}
  \else
%    \end{macrocode}
%    If there was a |\columnbreak| on the very last line all material
%    will have been moved to the |\colbreak@box|. Thus the the galley
%    will be empty and no output routine gets called so that the text
%    is lost. To avoid this problem (though unlikely) we check if the
%    current galley is empty and the |\colbreak@box| contains text and
%    if so return that to the galley. If the galley is non-empty any
%    material in  |\colbreak@box| is added in the output routine since
%    it needs to be put in front.
% \changes{v1.6d}{2003/04/17}{Catch problem with \cs{columnbreak} in
%    last line}
%    \begin{macrocode}
    \ifdim\pagegoal=\maxdimen
      \ifvoid\colbreak@box\else
        \mult@info\@ne{Re-adding forced
                 break(s) for splitting}%
        \unvbox\colbreak@box\fi
    \fi
%    \end{macrocode}
%    If we are in an unrestricted \mc{} environment we end the current
%    paragraph above with |\par| but this isn't sufficient since \TeX{}s
%    {\it page\_builder} will not totally empty the contribution
%    list.\footnote{This once caused a puzzling bug where some of the
%    material was balanced twice, resulting in some overprints.  The
%    reason was the \cs{eject} which was placed at the end of
%    the contribution list. Then the {\it page\_builder} was called
%    (an explicit \cs{penalty} will empty the contribution
%    list), but the line with the \cs{eject} didn't fit onto
%    the current page.  It was then reconsidered after the output
%    routine had ended, causing a second break after one line.}
%    Therefore we must also add an explicit |\penalty|.  Now the
%    contribution list will be emptied and, if its material doesn't
%    all fit onto the current page then the output routine will be
%    called before we change it.
% \changes{v1.3b}{1990/10/09}{Do \cs{penalty} with \cs{addpenalty}}
% \changes{v1.4e}{1992/03/16}{But ignore \cs{@nobreak} in \cs{addpenalty}}
% \changes{v1.5c}{1993/04/18}{Again use \cs{penalty}}
%    At this point we need to use |\penalty| not |\addpenalty| to
%    ensure that a) the recent contributions are emptied and b) that
%    the very last item on the main vertical list is a valid break
%    point so that \TeX{} breaks the page in case it is overfull.
%    \begin{macrocode}
    \penalty\z@
%    \end{macrocode}
%    Now it's safe to call the output routine in order to balance
%    the columns. We do this by calling it with a special penalty.
% \changes{v1.8r}{2018/01/28}{Use special penalty to signal end of environment}
%    \begin{macrocode}
    \penalty-\@Mvi
%    \end{macrocode}
%    If the \mc{} environment body was completely empty or if a
%    multi-page \mc{} just ends at a page boundary we have the unusual
%    case that the |\eject| will have no effect (since the main
%    vertical list is empty)---thus no output routine is called at
%    all. As a result the material preceding the \mc{} (stored in
%    |\partial@page| will get lost if we don't put this back by hand.
% \changes{v1.4m}{1992/09/04}{Check \cs{partial@page} being emptied}
%    \begin{macrocode}
    \ifvbox\partial@page
         \unvbox\partial@page\fi
%    \end{macrocode}
%    After the output routine has acted we restore
%    the kept marks to their initial value.
%    \begin{macrocode}
     \global\let\kept@firstmark\@empty
     \global\let\kept@botmark\@empty
%<*marktrace>
     \mult@info\tw@
       {Make kept marks empty}%
%</marktrace>
  \fi
%    \end{macrocode}
%    The output routine above will take care of the |\vsize| and
%    reinsert the balanced columns, etc. But it can't reinsert the
%    |\footnotes| because we first have to restore the
%    |\footins| parameter since we are returning to one column
%    mode. This will be done in the next line of code; we simply close
%    the group started in |\multicols|.
%
%    To fix an obscure bug which is the result of the current
%    definition of the |\begin| \ldots\ |\end| macros, we check that
%    we are still (logically speaking) in the \mc{} environment. If,
%    for example, we forget to close some environment inside the
%    \mc{} environment, the following |\endgroup| would be
%    incorrectly considered to be the closing of this environment.
% \changes{v1.3c}{1991/03/14}{Check closing env.}
%    \begin{macrocode}
  \@checkend{multicols}%
  \endgroup
%    \end{macrocode}
%    We also set the `{\sf unbalance}' counter to its default. This is
%    done globally since \LaTeX{} counters are always changed this
%    way.\footnote{Actually, we are still in a group started by the
%    \cs{begin} macro, so \cs{global} must be used
%    anyway.}
%    \begin{macrocode}
  \global\c@unbalance\z@
%    \end{macrocode}
%    Now it's time to return any footnotes if we are in unrestricted
%    mode. In boxed mode footnotes are kept inside, but in that case
%    we have to write another column status into the \texttt{.aux}
%    file to support \cs{docolaction} in case we have nested
%    environments.
% \changes{v1.8s}{2018/04/20}{Support for \cs{docolaction} (issue/39)}
%    \begin{macrocode}
  \if@boxedmulticols
    \mc@col@status@write
  \else
    \reinsert@footnotes
%    \end{macrocode}
%    We also take a look at the amount of free space on the current
%    page to see if it's time for a page break.  The vertical space
%    added thereafter will vanish if |\enough@room| starts a new
%    page.
%
%    But there is one catch. If the |\end{multicols}| is at the top of
%    which can happen if there is a break point just before it (such
%    as end ending environment) which was chosen. In that case we
%    would do the next page using the internal |\vsize| for multicol
%    collection which is a disaster. So we better catch this
%    case. Fortunately we can detect it by looking at |\pagegoal|.
% \changes{v1.5x}{2000/05/05}{Detect and fix problem if a multicols
%    ends at the top of a page}
%    \begin{macrocode}
    \ifdim \pagegoal=\maxdimen
      \global\vsize\@colroom
    \else
      \enough@room\postmulticols
    \fi
  \fi
  \addvspace\multicolsep
%    \end{macrocode}
%    There is one more thing to do: the balanced result of the
%   environment is supposed to have a |\prevdepth| of zero as we
%   backed up by its real prevdepth within |\page@sofar|. However if
%   the balancing happened in the output routine then \TeX{} reverts
%   to the |\prevdepth| that was current before the OR once the OR has
%   finished. In short |\prevdepth| is something you can't set
%   globally it is alway local to the current list being built. Thus
%   we need to set it back to zero here to avoid incorrect spacing.
% \changes{v1.8h}{2014/09/12}{Set \cs{prevdepdth} for current vlist
%   when returning from multicols environment}
%    \begin{macrocode}
 \prevdepth\z@
%    \end{macrocode}
%    If statistics are required we finally report that we have
%    finished everything.
%    \begin{macrocode}
  \mult@info\z@
     {Ending environment
             \if@boxedmulticols
                \space(boxed mode)\fi
              }}
%    \end{macrocode}
% \end{macro}
%
%
% \begin{macro}{\c@unbalance}
% \SpecialMainIndex{\c@collectmore}
% \SpecialMainIndex{\col@number}
% \SpecialMainIndex{\doublecol@number}
% \SpecialMainIndex{\premulticols}
% \SpecialMainIndex{\multicoltolerance}
% \SpecialMainIndex{\multicolpretolerance}
% \SpecialMainIndex{\page@free}
% \SpecialMainIndex{\premulticols}
% \SpecialMainIndex{\postmulticols}
% \SpecialMainIndex{\multicolsep}
% \SpecialMainIndex{\multicolbaselineskip}
% \SpecialMainIndex{\partial@page}
%    Let us end this section by allocating all the registers used so
%    far.
%    \begin{macrocode}
\newcount\c@unbalance
\newcount\c@collectmore
%    \end{macrocode}
%    In the new \LaTeX{} release |\col@number| is already allocated by
%    the kernel, so we don't allocate it again.
%    \begin{macrocode}
%\newcount\col@number
\newcount\doublecol@number
\newcount\multicoltolerance
\newcount\multicolpretolerance
\newdimen\full@width
\newdimen\page@free
\newdimen\premulticols
\newdimen\postmulticols
\newskip\multicolsep
\newskip\multicolbaselineskip
\newbox\partial@page
\newbox\last@line
%    \end{macrocode}
%    And here are their default values:
%    \begin{macrocode}
\c@unbalance   = 0
\c@collectmore = 0
%    \end{macrocode}
%    To allow checking whether some macro is used within the \mc{}
%    environment the counter |\col@number| gets a default of |1|
%    outside the the environment.
% \changes{v1.3d}{1991/10/23}{\cs{col@number} set to one}
% \changes{v1.8o}{2016/02/08}{\cs{col@number} already initialized in the 
%    kernel, so not initializing it in the package 
%    in case the document is in two-column (pr/4435)}
%    \begin{macrocode}
%\col@number = 1
\multicoltolerance = 9999
\multicolpretolerance = -1
\premulticols = 50pt
\postmulticols= 20pt
\multicolsep = 12pt plus 4pt minus 3pt
\multicolbaselineskip=0pt
%    \end{macrocode}
% \end{macro}
%
% \end{multicols}
%
% \begin{multicols}{2}[\subsection{The output routines}]
%
% We first start with some simple macros. When typesetting the page we
% save the columns either in the box registers 0, 2, 4,\ldots\
% (locally) or 1, 3, 5,\ldots\ (globally). This is \PlainTeX{} policy
% to avoid an overflow of the save stack.
%
% \begin{macro}{\process@cols}
%    Therefore we define a |\process@cols| macro to help us in
%    using these registers in the output routines below. It has two
%    arguments: the first one is a number; the second one is the
%    processing information.  It loops starting with |\count@=#1|
%    (|\count@| is a scratch register defined in \PlainTeX),
%    processes argument |#2|, adds two to |\count@|,
%    processes argument |#2| again, etc.\ until |\count@| is
%    higher than |\doublecol@number|.  It might be easier to
%    understand it through an example, so we define it now and
%    explain its usage afterwards.
%    \begin{macrocode}
\def\process@cols#1#2{\count@#1\relax
     \loop
%<*debug>
     \typeout{Looking at box \the\count@}
%</debug>
      #2%
     \advance\count@\tw@
     \ifnum\count@<\doublecol@number
   \repeat}
%    \end{macrocode}
% \end{macro}
%
%
% \begin{macro}{\page@sofar}
%    We now define |\page@sofar| to give an example of the
%    |\process@cols| macro. |\page@sofar| should output everything
%    prepared by the balancing routine |\balance@columns|.
%    \begin{macrocode}
\def\page@sofar{%
%    \end{macrocode}
%    |\balance@columns| prepares its output in the even numbered
%    scratch box registers.
%    Now we output the columns gathered assuming that they are saved
%    in the box registers 2 (left column), 4 (second column), \ldots\
%    However, the last column (i.e.\ the right-most) should be saved in
%    box register 0.\footnote{You will see the reason for this numbering
%                             when we look at the output routines
%                             \cs{multi@column@out} and
%                             \cs{balance@columns@out}.}
%    First we ensure that the columns have equal width. We use
%    |\process@cols| for this purpose, starting with
%    $|\count@|=|\mult@rightbox|$. Therefore |\count@| loops through
%    |\mult@rightbox|, $|\mult@rightbox| + 2$,\ldots
%                                  (to |\doublecol@number|).
% \changes{v1.5a}{1992/11/04}{New box mechanism}
%    \begin{macrocode}
   \process@cols\mult@rightbox
%    \end{macrocode}
%    We have to check if the box in question is void, because the
%    operation |\wd|\meta{number} on a void box will \emph{not} change
%    its dimension (sigh).
% \changes{v1.5h}{1994/08/26}{Check for void boxes}
% \changes{v1.5i}{1994/10/02}{But don't remove original code.}
%    \begin{macrocode}
       {\ifvoid\count@
          \setbox\count@\hbox to\hsize{}%
        \else
          \wd\count@\hsize
        \fi}%
%    \end{macrocode}
%    Now we give some tracing information.
% \changes{v1.4l}{1992/08/17}{use \cs{multicol@leftmargin} instead of
%                           \cs{@totalleftmargin}}
% \changes{v1.6b}{2003/03/15}{Different info display}
%    \begin{macrocode}
   \count@\col@number \advance\count@\m@ne
   \mult@info\z@
    {Column spec: \the\full@width\space = indent
                  + columns + sep =\MessageBreak
        \the\multicol@leftmargin\space
        + \the\col@number\space
        x \the\hsize\space
        + \the\count@\space
        x \the\columnsep
     }%
%    \end{macrocode}
%    At this point we should always be in vertical mode.
%    \begin{macrocode}
\ifvmode\else\errmessage{Multicol Error}\fi
%    \end{macrocode}
%    Now we put all columns together in an |\hbox| of width
%    |\full@width| (shifting it by |\multicol@leftmargin| to the right
%    so that it will be placed correctly if we are within a list
%    environment)
%    and separating the columns with a rule if desired.
% \changes{v1.5a}{1992/11/04}{New box mechanism}
% \changes{v1.5z}{2000/07/10}{Ensure that column rule has always
%                             \cs{normalcolor}}
% \changes{v1.6f}{2006/09/19}{Make the color of the rule a hook}
% \changes{v1.4l}{1992/08/17}{use \cs{multicol@leftmargin} instead of
%                           \cs{@totalleftmargin}}
%
%    The box containing the columns has  a large height and thus will
%    always result in using |\lineskip| if the normal |\baselineskip|
%    calculations are used. We therefore better cancel that process.
% \changes{v1.5z1}{2003/02/17}{Suppress interline glue at this point}
%    \begin{macrocode}
   \nointerlineskip
%    \end{macrocode}
%    As mentioned earlier we want to have the reference point of the
%    box we put on the page being at the baseline of the last line of
%    the columns but we also want to ensure that the box has no depth
%    so that any following skip is automatically starting from that
%    baseline.
%    We achieve this by recording the depths of all columns and then
%    finally backing up by the maximum. (perhaps a simpler method
%    would be to assemble the box in a register and set the depth of
%    that box to zero (not checked).
%
%    We need a global scratch register for this; using standard \TeX{}
%    conventions we choose |\dimen2| and initialize it with the depth
%    of the character ``p'' since that is one of the depths that
%    compete for the maximum.
% \changes{v1.6a}{2003/03/15}{Preparing for adjusting \cs{prevdepth}}
%    \begin{macrocode}
   \setbox\z@\hbox{p}\global\dimen\tw@\dp\z@
   \moveright\multicol@leftmargin
    \hbox to\full@width{%
%    \end{macrocode}
%    If the document is written in a language that is typeset right-to-left
%    then, of course, the multicol columns should be also typeset
%    right-to-left. To support this we call |\mc@align@columns| which
%    with execute different code depending on the typesetting direction.
% \changes{v1.7b}{2011/12/18}{RL language support fixed}
%    \begin{macrocode}
       \mc@align@columns
%    \end{macrocode}
%    The depths of the columns depend on their last lines. To ensure
%    that we will always get a similar look as far as the rules are
%    concerned we force the depth to be at least the depth of a
%    letter~`p' (which is what we set |\dimen2| to above).
%    \begin{macrocode}
       \rlap{\phantom p}%
    }%
%    \end{macrocode}
%    The processed material might consist of a last line with a
%    descender in which case the |\prevdepth| will be
%    non-zero. However, this material is getting reformatted now so
%    that this value is likely to be wrong. We therefore normalize the
%    situation by pretending that the depth is zero. However, if |\page@sofar|
%   is being called inside the OR then setting |\prevdepth| here has
%   no longlasting effect, we therefore have to repeat this once we
%   return to the main vertical list. Here we set it only for those
%   cases where the command is used within a list and then followed by
%   something else.
% \changes{v1.8g}{2014/08/24}{Resetting \cs{prevdepth} in the right place}
%    \begin{macrocode}
   \prevdepth\z@
%    \end{macrocode}
%    Now after typesetting the box we back up to its baseline by using
%    the value stored in |\dimen2| (which will hold the largest depth
%    found on any column).
% \changes{v1.8g}{2003/03/15}{Now adjusting \cs{prevdepth}}
%    \begin{macrocode}
 \kern-\dimen\tw@
%    \end{macrocode}
%    However, in case one of the columns was unusually deep \TeX{} may
%    have tried some corrective actions in which case backing up by
%    the saved value will not bring us back to the baseline. A good
%    indication for this is a depth of |\maxdepth| though it is not an
%    absolute proof. If the option \texttt{grid} is used
%    |\mc@gridwarn| will expand to this, otherwise to |\maxdimen| in
%    which case this warning will not show up.
% \changes{v1.8g}{2014/08/24}{Warn if value is exceeded not when equal}
%    \begin{macrocode}
 \ifdim\dimen\tw@ > \mc@gridwarn
   \PackageWarning{multicol}%
     {Very deep columns!\MessageBreak
      Grid alignment might be broken}%
 \fi
}
%    \end{macrocode}
% \end{macro}
%
% \begin{macro}{\columnseprulecolor}
% \changes{v1.6f}{2006/09/19}{Make the color of the rule a hook}
%    By default the vertical rule between columns will be in |\normalcolor|.
%    \begin{macrocode}
\def\columnseprulecolor{\normalcolor}
%    \end{macrocode}
% \end{macro}
%
% \begin{macro}{\reinsert@footnotes}
%    Before we tackle the bigger output routines we define just one
%    more macro which will help us to find our way through the
%    mysteries later. |\reinsert@footnotes| will do what its name
%    indicates: it reinserts the footnotes present in
%    |\footinbox| so that they will be reprocessed by \TeX{}'s
%    {\it page\_builder}.
%
%    Instead of actually reinserting the footnotes we insert an empty
%    footnote. This will trigger insertion mechanism as well and since
%    the old footnotes are still in their box and we are on a fresh page
%    |\skip| |footins| should be correctly taken into account.
% \changes{v1.3c}{1990/03/03}{\cs{unbox}ing avoided.}
%    \begin{macrocode}
\def\reinsert@footnotes{\ifvoid\footins\else
         \insert\footins{}\fi}
%    \end{macrocode}
% \end{macro}
%
%
% \begin{macro}{\vfilmaxdepth}
%    This curious definition is used as the space at the bottom of a
%   column if we implement |\raggedcolumns|. Normlly one only appends
%   |\vfill| in that case but this is actually wrong for columns that
%   are more or less full: by adding a glue at the bottom such a
%   column doesn't have any depth any more but without it the material
%   would be allowed a depth of |\maxdepth|. So we allow shrinking by
%   that amount. This only makes a difference if the box would
%   otherwise become overfull and shrinking never exceeds the
%   specified value, so we should be fine. 
% \changes{v1.8h}{2014/09/12}{Macro added  (pr/4395)}
% \changes{v1.8j}{2015/03/07}{Use only `0.0001fil' for stretching}
%    \begin{macrocode}
\def\vfilmaxdepth{\vskip \z@ \@plus .0001fil 
                           \@minus \maxdepth}
%    \end{macrocode}
% \end{macro}
%
%
% \begin{macro}{\multi@column@out}
%    Now we can't postpone the difficulties any longer.  The
%    |\multi@column@out| routine will be called in two situations.
%    Either the page is full (i.e., we have collected enough material
%    to generate all the required columns) or a float or marginpar or
%    a |\clearpage| is
%    sensed.  In the latter case the |\outputpenalty| is less
%    than $-10000$, otherwise the penalty which triggered the output
%    routine is higher.  Therefore it's easy to distinguish both
%    cases: we simply test this register.
% \changes{v1.5c}{1993/04/18}{Support \cs{clearpage}}
%    \begin{macrocode}
\def\multi@column@out{%
   \ifnum\outputpenalty <-\@M
%    \end{macrocode}
%    If this was a |\clearpage|, a float or a marginpar we call
%    |\speci@ls|
%    \begin{macrocode}
    \speci@ls \else
%    \end{macrocode}
%    otherwise we construct the final page.
% \changes{v1.5u}{1999/05/25}{Support \cs{columnbreak}}
% \changes{v1.5v}{1999/07/18}{Added debug statements for
%                             column break support}
%    For the next block of code see comments in section~\ref{sec:colbreak}.
%    \begin{macrocode}
    \ifvoid\colbreak@box\else
      \mult@info\@ne{Re-adding forced
                break(s) for splitting}%
      \setbox\@cclv\vbox{%
       \unvbox\colbreak@box
       \penalty-\@Mv
       \unvbox\@cclv}%
    \fi
%    \end{macrocode}
%    Let us now consider the
%    normal case. We have to |\vsplit| the columns from the
%    accumulated material in box 255.  Therefore we first assign
%    appropriate values to |\splittopskip| and |\splitmaxdepth|.
%    \begin{macrocode}
   \splittopskip\topskip
   \splitmaxdepth\maxdepth
%    \end{macrocode}
%    We also need to restrict |\boxmaxdepth| so that re-boxing is not
%    generating boxes with arbitrary depth.   
% \changes{v1.8j}{2015/03/07}{Set \cs{boxmaxdepth}}
%    \begin{macrocode}
   \boxmaxdepth\maxdepth
%    \end{macrocode}
%    Then we calculate the current column height (in |\dimen@|).
%    Note that the height of |\partial@page| is already
%    subtracted from |\@colroom| so we can use its value as a
%    starter.
%    \begin{macrocode}
   \dimen@\@colroom
%    \end{macrocode}
%    But we must also subtract the space occupied by footnotes on the
%    current page. Note that we first have to reset the skip register
%    to its normal value.
%    Again, the actual action is carried out in a utility macro, so that
%    other applications can modify it.
% \changes{v1.5?}{1994/?/?}{Use \cs{leave@mult@footins}}
%    \begin{macrocode}
   \divide\skip\footins\col@number
   \ifvoid\footins \else
      \leave@mult@footins
   \fi
%    \end{macrocode}
%    And there is one more adjustment that we have to make: if the
%    user has issue a |\enlargethispage| command then the height the
%    |\@kludgeins| box will be the negation of the size by which the
%    page should be enlarged. If the star form of this command has
%    been used then we also need to shrink the resulting column. As we
%    don't know whether or not shrinking is already generally
%    requested with save the current value of |\ifshr@king| and
%    restore it afterwards.
% \changes{v1.8a}{2011/12/20}{Support for \cs{enlargethispage}}
%    \begin{macrocode}
   \let\ifshr@kingsaved\ifshr@king
   \ifvbox \@kludgeins
     \advance \dimen@ -\ht\@kludgeins
%    \end{macrocode}
%    The star form of  |\enlargethispage| makes the width of the box
%    greater than zero (sneaky isn't it?).
%    \begin{macrocode}
     \ifdim \wd\@kludgeins>\z@
        \shr@nkingtrue
     \fi
   \fi
%    \end{macrocode}
%    Now we are able to |\vsplit| off all but the last column.
%    Recall that these columns should be saved in the box registers 2,
%    4,\ldots\ (plus offset).
% \changes{v1.5a}{1992/11/04}{New box mechanism}
%    \begin{macrocode}
   \process@cols\mult@gfirstbox{%
        \setbox\count@
            \vsplit\@cclv to\dimen@
%    \end{macrocode}
%    After splitting we update the kept marks.
%    \begin{macrocode}
            \set@keptmarks
%    \end{macrocode}
%    If |\raggedcolumns| is in force we add a |vfill| at the bottom by
%    unboxing the split box.
%    But we need to unbox anyway to ensure that at the end of the box
%    we do not have unwanted space. This can sneak in in certain
%    situations, for example, if two lists follow each other and we
%    break between them. While sich space is usually zero it still has
%    an effect because it hides depth of the last line in the column
%    and that will result in incorrect placement.
% \changes{v1.3c}{1990/03/03}{\cs{unbox}ing avoided.}
% \changes{v1.8b}{2013/08/28}{And 20odd years later conclude that
%         this was wrong and unboxing is always needed.}
% \changes{v1.8b}{2013/08/28}{Remove discardable items at the
%                             end of split boxes}
% \changes{v1.8h}{2014/09/12}{Use \cs{vfilmaxdepth}}
% \changes{v1.8k}{2015/03/22}{\cs{remove@discardable@items} removed}
%    \begin{macrocode}
            \setbox\count@
                 \vbox to\dimen@
                  {\unvbox\count@
                   \ifshr@nking\vfilmaxdepth\fi}%
           }%
%    \end{macrocode}
%    Then the last column follows.
% \changes{v1.5a}{1992/11/04}{New box mechanism}
% \changes{v1.8b}{2013/08/28}{Remove discardable items at the
%                             end of split boxes}
% \changes{v1.8h}{2014/09/12}{Use \cs{vfilmaxdepth}}
% \changes{v1.8k}{2015/03/22}{\cs{remove@discardable@items} removed}
%    \begin{macrocode}
   \setbox\mult@rightbox
       \vsplit\@cclv to\dimen@
   \set@keptmarks
   \setbox\mult@rightbox\vbox to\dimen@
          {\unvbox\mult@rightbox
           \ifshr@nking\vfilmaxdepth\fi}%
%    \end{macrocode}
%    Now that we are done with the boxes, we restored the current
%    setting for shrinking in case it got changed:
%    \begin{macrocode}
   \let\ifshr@king\ifshr@kingsaved
%    \end{macrocode}
%    Having done this we hope that box 255 is emptied. If not, we
%    reinsert its contents.
% \changes{v1.8a}{2011/12/20}{Only re-add output penalty if it was
%    explicitly set}
%    \begin{macrocode}
   \ifvoid\@cclv \else
       \unvbox\@cclv
       \ifnum\outputpenalty=\@M
       \else
          \penalty\outputpenalty
       \fi
%    \end{macrocode}
%    In this case a footnote that happens to fall into the leftover
%    bit will be typeset on the wrong page. Therefore we warn the user
%    if the current page contains footnotes. The older versions of
%    \mc{} produced this warning regardless of whether or not
%    footnotes were present, resulting in many unnecessary warnings.
% \changes{v1.3c}{1991/02/17}{Check if footnotes are actually present
%                           before issuing a warning.}
%    \begin{macrocode}
       \ifvoid\footins\else
         \PackageWarning{multicol}%
          {I moved some lines to
           the next page.\MessageBreak
           Footnotes on page
           \thepage\space might be wrong}%
       \fi
%    \end{macrocode}
%    If the `{\sf tracingmulticols}' counter is 4 or higher we also
%    add a rule.
%    \begin{macrocode}
       \ifnum \c@tracingmulticols>\thr@@
                    \hrule\allowbreak \fi
   \fi
%    \end{macrocode}
%    To get a correct marks for the current page we have to (locally)
%    redefine |\firstmark| and |\botmark|.
%    If |\kept@firstmark| is non-empty then |\kept@botmark| must be
%    non-empty too so we can use their values. Otherwise we use the
%    value of |\kept@topmark| which was first initialized when we
%    gathered the |\partical@page| and later on was updated to the
%    |\botmark| for the preceding page.
%
% \changes{v1.4a}{1992/02/14}{\cs{botmark} set to \cs{splitbotmark}}
%    \begin{macrocode}
   \ifx\@empty\kept@firstmark
      \let\firstmark\kept@topmark
      \let\botmark\kept@topmark
   \else
      \let\firstmark\kept@firstmark
      \let\botmark\kept@botmark
   \fi
%    \end{macrocode}
%    We also initalize |\topmark| with |\kept@topmark|. This will make
%    this mark okay for all middle pages of the \mc{} environment.
% \changes{v1.5d}{1993/09/15}{reinit \cs{topmark}}
%    \begin{macrocode}
   \let\topmark\kept@topmark
%<*marktrace>
   \mult@info\tw@
        {Use kept top mark:\MessageBreak
          \meaning\kept@topmark
         \MessageBreak
         Use kept first mark:\MessageBreak
          \meaning\kept@firstmark
        \MessageBreak
         Use kept bot mark:\MessageBreak
          \meaning\kept@botmark
        \MessageBreak
         Produce first mark:\MessageBreak
          \meaning\firstmark
        \MessageBreak
        Produce bot mark:\MessageBreak
          \meaning\botmark
         \@gobbletwo}%
%</marktrace>
%    \end{macrocode}
%    With a little more effort we could have done better. If we had,
%    for example, recorded the shrinkage of the material in
%    |\partial@page| it would be now possible to try higher
%    values for |\dimen@| (i.e.\ the column height) to overcome
%    the problem with the nonempty box 255. But this would make the
%    code even more complex so I skipped it in the current
%    implementation.
%
%    Now we use \LaTeX{}'s standard output
%    mechanism.\footnote{This will produce a lot of overhead since both
%                       output routines are held in memory. The correct
%                       solution would be to redesign the whole output
%                       routine used in \LaTeX.}
%    Admittedly this is a funny way to do it.
%
%    Within the OR |\boxmaxdepth| needs to be unrestricted so we set it back
%    now as it was changed above.
% \changes{v1.8p}{2016/04/07}{Reset \cs{boxmaxdepth}} 
%    \begin{macrocode}
   \boxmaxdepth\maxdimen
   \setbox\@cclv\vbox{\unvbox\partial@page
                      \page@sofar}%
%    \end{macrocode}
%    The macro |\@makecol| adds all floats assigned for the current
%    page to this page.  |\@outputpage| ships out the resulting box.
%    Note that it is just possible that such floats are present even
%    if we do not allow any inside a \mc{} environment.
%    \begin{macrocode}
   \@makecol\@outputpage
%    \end{macrocode}
%    After the page is shipped out we have to prepare the kept marks
%    for the following page. |\kept@firstmark| and |\kept@botmark|
%    reinitialized by setting them to |\@empty|.  The value of
%    |\botmark| is then assigned to |\kept@topmark|.
% \changes{v1.4g}{1992/06/03}{Only change \cs{kept@topmark} if
%                            \cs{kept@botmark} non-empty}
% \changes{v1.4i}{1992/06/18}{Set \cs{kept@topmark} to \cs{botmark}}
%    \begin{macrocode}
     \global\let\kept@topmark\botmark
     \global\let\kept@firstmark\@empty
     \global\let\kept@botmark\@empty
%<*marktrace>
     \mult@info\tw@
        {(Re)Init top mark:\MessageBreak
         \meaning\kept@topmark
         \@gobbletwo}%
%</marktrace>
%    \end{macrocode}
%    Now we reset |\@colroom| to |\@colht| which is \LaTeX's
%    saved value of |\textheight|.
%    We also have to reset the recorded position of the last
%   |\marginpar| as well as the recorded size of intext floats 
%    as we are now on a new page.
% \changes{v1.8d}{2014/04/23}{Reset \cs{@mparbottom} after page finishes}
% \changes{v1.8n}{2015/08/19}{Reset \cs{@textfloatsheight} after page finishes}
%    \begin{macrocode}
   \global\@colroom\@colht
   \global \@mparbottom \z@
   \global \@textfloatsheight \z@
%    \end{macrocode}
%    Then we process deferred floats waiting for their chance to be
%    placed on the next page.
%    \begin{macrocode}
   \process@deferreds
   \@whilesw\if@fcolmade\fi{\@outputpage
      \global\@colroom\@colht
      \process@deferreds}%
%    \end{macrocode}
%    If the user is interested in statistics we inform him about the
%    amount of space reserved for floats.
%    \begin{macrocode}
   \mult@info\@ne
     {Colroom:\MessageBreak
      \the\@colht\space
              after float space removed
              = \the\@colroom \@gobble}%
%    \end{macrocode}
%    Having done all this we must prepare to tackle the next page.
%    Therefore we assign a new value to |\vsize|.  New, because
%    |\partial@page| is now empty and |\@colroom| might be
%    reduced by the space reserved for floats.
% \changes{v1.4p}{1992/11/26}{Use different \cs{vsize} setting}
%    \begin{macrocode}
    \set@mult@vsize \global
%    \end{macrocode}
%    The |\footins| skip register will be adjusted when the output
%    group is closed.
% \changes{v1.3c}{1991/03/03}{Unnecessary code removed}
%    \begin{macrocode}
  \fi}
%    \end{macrocode}
% \end{macro}
%
%
%
% \begin{macro}{\leave@mult@footins}
%    This macro is used to subtract the amount of space
%    occupied by footnotes for the current space from the
%    space available for the current column. The space current column
%    is stored in |\dimen@|. See above for the description of the default
%    action.
% \changes{v1.5?}{1994/?/?}{Macro added}
%    \begin{macrocode}
\def\leave@mult@footins{%
   \advance\dimen@-\skip\footins
   \advance\dimen@-\ht\footins
}
%    \end{macrocode}
% \end{macro}
%
%
%
% \begin{macro}{\speci@ls}
%    We left out two macros: |\process@deferreds| and
%    |\speci@ls|.
% \changes{v1.5c}{1993/04/18}{Support \cs{clearpage}}
%    \begin{macrocode}
\def\speci@ls{%
 \ifnum\outputpenalty <-\@Mi
%    \end{macrocode}
%    If the document ends in the middle of a multicols environment, e.g.,
%    if the user forgot the |\end{multicols}|, \TeX{} adds a very
%    negative penalty to the end of the galley which is intended to signal
%    the output routine that it is time to prepare for shipping out
%    everything remaining. Since inside multicols the output routine of
%    \LaTeX{} is disabled sometimes we better check for this case: if we
%    find a very negative penalty we produce an error message and run
%    the default output routine for this case.
% \changes{v1.5s}{1998/09/10}{check for \cs{stop} penalty pr/2873}
%    \begin{macrocode}
  \ifnum \outputpenalty<-\@MM
   \PackageError{multicol}{Document end
           inside multicols environment}\@ehd
   \@specialoutput
  \else
%    \end{macrocode}
% \changes{v1.5u}{1999/05/25}{Support \cs{columnbreak}}
% \changes{v1.5v}{1999/07/18}{Added debug statements for
%                             column break support}
%    For the next block of code see comments in section~\ref{sec:colbreak}.
%    \begin{macrocode}
     \ifnum\outputpenalty = -\@Mv
         \mult@info\@ne{Forced column
                        break seen}%
         \global\advance\vsize-\pagetotal
         \global\setbox\colbreak@box
           \vbox{%
              \ifvoid\colbreak@box
              \else
                \unvbox\colbreak@box
                \penalty-\@Mv
              \fi
%    \end{macrocode}
%    As this is the place of a forced break we now remove vertical
%   white space just in front of it (or some of it at least) as it is
%   quite likely that the break is not exactly in the right place,
%   e.g., after a display environment (if LaTeX would break here by
%   its own it would break before the space following the display).
%
%   Thus we rebox box 255 once (using |\maxdepth| and calling
%   |\remove@discardable@items| inside). The depth of 255 will then
%   give us the depth the box would have had if it would have been a
%   natural break. We then unbox 255 to get it into the
%   |\colbreak@box| and then back up by this depth. This will position
%   the bottom of the box at its natural baseline which is useful for
%   balancing later on.
% \changes{v1.8k}{2015/03/21}{Remove discarable items just before a
%   forced break}
%    \begin{macrocode}
              \boxmaxdepth\maxdepth
              \setbox\@cclv\vbox{%
                 \unvbox\@cclv
                 \remove@discardable@items}%
              \dimen@\dp\@cclv
              \unvbox\@cclv
              \kern-\dimen@
         }%
         \reinsert@footnotes
     \else
%    \end{macrocode}
%    Another special case is reaching the end of the \mc{}
%    environment which is signalled by |-\@Mvi|.
% \changes{v1.8r}{2018/01/28}{Handling end of env through special penalty}
%    \begin{macrocode}
       \ifnum\outputpenalty = -\@Mvi
         \mult@info\@ne{End penalty of multicols seen}%
%    \end{macrocode}
%    If we are at this point then have have to run the balancing code
%    (which was previously its own output routine). First we pretend
%    that we had a normal forced breakpoint and then call
%    |\balance@column@out|. The latter may be let to |\multi@column@out|
%    if we are inside \texttt{multicols*} in which case we would get a loop
%    if the |\outputpenalty| is not changed---this could be cleaned up
%    in a better way; basically it is like this, because of the older
%    code was using different ORs and I simply reused most of it.
%    \begin{macrocode}
         \outputpenalty\@M   % pretend we had a natural forced break
         \balance@columns@out
       \else
%    \end{macrocode}
%    If we encounter a float or a marginpar in the current
%    implementation we simply warn the user that this is not allowed.
%    Then we reinsert the page and its footnotes.
%    \begin{macrocode}
        \PackageWarningNoLine{multicol}%
           {Floats and marginpars not
            allowed inside `multicols'
            environment!}%
        \unvbox\@cclv\reinsert@footnotes
%    \end{macrocode}
%    Additionally we empty the |\@currlist| to avoid later error
%    messages when the \LaTeX{} output routine is again in force.
%    But first we have to place the boxes back onto the
%    |\@freelist|. (|\@elt|s default is |\relax| so
%    this is possible with |\xdef|.)
% \changes{v1.2a}{1990/02/05}{Float boxes freed.}
%    \begin{macrocode}
        \xdef\@freelist{\@freelist\@currlist}%
        \gdef\@currlist{}%
      \fi
    \fi
  \fi
%    \end{macrocode}
%    If the penalty is $-10001$ it will come from a |\clearpage| and
%    we will execute |\@doclearpage| to get rid of any deferred
%    floats.
%    \begin{macrocode}
 \else \@doclearpage \fi
}
%    \end{macrocode}
% \end{macro}
%
% \begin{macro}{\process@deferreds}
%    |\process@deferreds| is a simplified version of \LaTeX{}'s
%    |\@startpage|.  We first call the macro
%    |\@floatplacement| to save the current user parameters in
%    internal registers.  Then we start a new group and save the
%    |\@deferlist| temporarily in the macro |\@tempb|.
%    \begin{macrocode}
\def\process@deferreds{%
   \@floatplacement
   \@tryfcolumn\@deferlist
   \if@fcolmade\else
     \begingroup
    \let\@tempb\@deferlist
%    \end{macrocode}
%    Our next action is to (globally) empty |\@deferlist| and
%    assign a new meaning to |\@elt|.  Here |\@scolelt| is a
%    macro that looks at the boxes in a list to decide whether they
%    should be placed on the next page (i.e.\ on |\@toplist| or
%    |\@botlist|) or should wait for further processing.
%    \begin{macrocode}
      \gdef\@deferlist{}%
      \let\@elt\@scolelt
%    \end{macrocode}
%    Now we call |\@tempb| which has the form
%    \begin{center}
%      |\@elt|\meta{box register}|\@elt|^^A
%                        \meta{box register}\ldots{}
%    \end{center}
%    So |\@elt| (i.e.\ |\@scolelt|) will distribute the
%    boxes to the three lists.
%    \begin{macrocode}
        \@tempb \endgroup
   \fi}
%    \end{macrocode}
% \end{macro}
%
%
%
% \begin{macro}{ifshr@nking}
% \begin{macro}{\raggedcolumns}
% \begin{macro}{\flushcolumns}
% \changes{v1.1a}{1989/09/20}{\cs{flushedcolumns} renamed to \cs{flushcolumns}.}
%    The |\raggedcolumns| and |\flushcolumns|
%    declarations are defined with the help of a new |\if...|
%    macro.
%    \begin{macrocode}
\newif\ifshr@nking
%    \end{macrocode}
%    The actual definitions are simple: we just switch to {\sf true}
%    or {\sf false} depending on the desired action. To avoid extra
%    spaces in the output we enclose these changes in
%    |\@bsphack|\ldots{}\allowbreak|\@esphack|.
%    \begin{macrocode}
\def\raggedcolumns{%
   \@bsphack\shr@nkingtrue\@esphack}
\def\flushcolumns{%
   \@bsphack\shr@nkingfalse\@esphack}
%    \end{macrocode}
% \end{macro}
% \end{macro}
% \end{macro}
%
%
% \begin{macro}{\balance@columns@out}
%    Now for the last part of the show: the column balancing output
%    routine.  Since this code is called with an explicit penalty
%    (|\eject|) there is no need to check for something special (eg
%    floats).  We start by balancing the material gathered.
%    \begin{macrocode}
\def\balance@columns@out{%
%    \end{macrocode}
%    For this we need to put the contents of box 255 into |\mult@box|.
% \changes{v1.5u}{1999/05/25}{Support \cs{columnbreak}}
% \changes{v1.5v}{1999/07/18}{Added debug statements for column break
%   support} For the next block of code see also comments in
%   section~\ref{sec:colbreak}.  All forced breaks except the last are
%   inside |\colbreak@box| so all we have to do is to concatenate this
%   box with box |\@cclv| and put a penalty inbetween. Here we test if
%   |\colbreak@box| is void so that the message is only generated if
%   we really add forced breaks and the penalty.
% \changes{v1.8k}{2015/03/21}{Use \cs{@Mv} and not \cs{break} in case
%   this forced break is not used on this page}
%    \begin{macrocode}
   \setbox\mult@box\vbox{%
       \ifvoid\colbreak@box\else
         \unvbox\colbreak@box
         \penalty-\@Mv
         \mult@info\@ne{Re-adding
           forced break(s) in balancing}%
       \fi
       \unvbox\@cclv
%    \end{macrocode}
%    The last column again is a forced break, so here we discard white
%   space as well as that is normally unwanted.
% \changes{v1.8k}{2015/03/21}{Add \cs{remove@discardable@items} at the
%   end of the last column when balancing.}
%    \begin{macrocode}
       \remove@discardable@items
   }%
   \balance@columns
%    \end{macrocode}
%    If during balancing the columns got too long the flag
%    |\iftoo@bad| is set to true.
% \changes{v1.8a}{2011/12/20}{Balancing concept improved}
%    \begin{macrocode}
   \iftoo@bad
     \mult@info\@ne
        {Balancing failed ...
         cut a normal page}%
%    \end{macrocode}
%    In that case we put the material back in box 255 so that we can
%    cut a normal page. The curious set of
%    |\vskip|s we add is necessary to  cancel out the |\splittopskip|
%    that got added for balancing.
%    \begin{macrocode}
     \setbox\@cclv\vbox
         {\vskip\topskip
          \vskip-\splittopskip
          \unvbox\mult@box
%    \end{macrocode}
%    We also have to re-add the end of environment penalty since after
%    this page we may want  balance the remaining material.
% \changes{v1.8r}{2018/01/28}{Re-add the final penalty}
%    \begin{macrocode}
          \penalty-\@Mvi
         }%
%    \end{macrocode}
%    We then call the standard multicol output routine which will
%    produce a normal page for us (remember we are still within the
%    OR so some part of the code in |\multi@column@out| is actually not
%    doing anything---perhaps this should be cleaned up at some point).
%    This also means that if there was an |\enlargethispage| present
%    it will apply to this page as |\multi@column@out| will look at
%    the status of |\@kludgeins|.
%    \begin{macrocode}
     \multi@column@out
%    \end{macrocode}
%    Because balancing made the columns too long we are sure that there
%    will be some material remaining which was put back onto the main
%    vertical list by |\multi@column@out|. This will also put the
%    explicit |\eject| penalty back so the current
%    |\balance@columns@out| output routine will be called again (so we
%    better do not add another penalty or else the OR will be called
%    twice and we may get scrambled results).
% \changes{v1.8k}{2015/03/21}{No additional penalty here}
%    \begin{macrocode}
   \else
%    \end{macrocode}
%    If the balancing went ok, we are in the position
%    to apply |\page@sofar|.  But first we have to set |\vsize| to a
%    value suitable for one column output.
%    \begin{macrocode}
     \global\vsize\@colroom
     \global\advance\vsize\ht\partial@page
%    \end{macrocode}
%    We also have to look at |\@kludgeins| and generate a new
%    |\insert| in case there was one present due to an
%    |\enlargethispage| command.
% \changes{v1.8a}{2011/12/20}{Support for \cs{enlargethispage}}
%    \begin{macrocode}
     \ifvbox\@kludgeins\insert\@kludgeins
                        {\unvbox\@kludgeins}\fi
%    \end{macrocode}
%    Then we |\unvbox| the |\partial@page| (which may be void if we
%    are not processing the first page of this \mc{} environment.
%    \begin{macrocode}
     \unvbox\partial@page
%    \end{macrocode}
%    Then we return the first and bottom mark and the gathered
%    material to the main vertical list.
%    \begin{macrocode}
     \return@nonemptymark{first}\kept@firstmark
     \return@nonemptymark{bot}\kept@botmark
     \page@sofar
%    \end{macrocode}
%    We need to add a penalty at this point which allows to break at
%    this point since calling the output routine may have removed the
%    only permissible break point thereby ``glueing'' any following
%    skip to the balanced box. In case there are any weird settings
%    for |\multicolsep| etc. this could produce funny results.
%  \changes{v1.5c}{1993/04/18}{added penalty at output routine exit}
%    \begin{macrocode}
     \penalty\z@
  \fi
}
%    \end{macrocode}
%    As we already know, reinserting of footnotes will be done in the
%    macro |\endmulticols|.
% \end{macro}
%
%
%  \begin{macro}{\balance@columns}
%    This macro now does the actual balancing.
%    \begin{macrocode}
\def\balance@columns{%
%    \end{macrocode}
%    We start by setting the kept marks by updating them with any
%    marks from this box. This has to be done \emph{before} we add a
%    penalty of $-10000$ to the top of the box, otherwise only an
%    empty box will be considered.
% \changes{v1.5h}{1994/08/26}{Get kept marks first}
%    \begin{macrocode}
   \get@keptmarks\mult@box
%    \end{macrocode}
%    We then continue by resetting trying to remove any discardable
%    stuff at the end of |\mult@box|. This is rather experimental.  We
%    also add a forced break point at the very beginning, so that we
%    can split the box to height zero later on, thereby adding a known
%    |\splittopskip| glue at the beginning.
% \changes{v1.8k}{2015/03/21}{\cs{remove@discardable@items} removed}
%    \begin{macrocode}
   \setbox\mult@box\vbox{%
        \penalty-\@M
        \unvbox\mult@box
        }%
%    \end{macrocode}
%    Then follow values assignments to get the |\vsplit|ting right.
%    We use the natural part of |\topskip| as the natural part for
%    |\splittopskip| and allow for a bit of undershoot and overshoot
%    by adding some stretch and shrink.
% \changes{v1.5?}{1994/?/?}{Allow columns to come out a bit long or short}
%    \begin{macrocode}
   \@tempdima\topskip
   \splittopskip\@tempdima
       \@plus\multicolundershoot
       \@minus\multicolovershoot
   \splitmaxdepth\maxdepth
%    \end{macrocode}
%   We also have to set |\boxmaxdepth| which normally allows to
%   build boxes with arbitrary depth, but as we are building text
%   columns we really want to restrict the depth. This is necessary as
%   we sometimes rebox the boxes generated by |\vsplit| and then the
%   restriction posed by |\splitmaxdepth|  gets lost.
% \changes{v1.8h}{2014/09/12}{All column boxes should obey
%   \cs{maxdepth}  (pr/4395)}
%    \begin{macrocode}
   \boxmaxdepth\maxdepth
%    \end{macrocode}
%    The next step is a bit tricky: when \TeX{} assembles material in
%    a box, the first line isn't preceded by interline glue, i.e.
%    there is no parameter like |\boxtopskip| in \TeX{}. This means
%    that the baseline of the first line in our box is at some
%    unpredictable point depending on the height of the largest
%    character in this line. But of course we want all columns to
%    align properly at the baselines of their first lines. For this
%    reason we have opened |\mult@box| with a |\penalty| {\sf -10000}.
%    This will now allow us to split off from |\mult@box| a tiny bit
%    (in fact nothing since the first possible break-point is the
%    first item in the box). The result is that |\splittopskip| is
%    inserted at the top of |\mult@box| which is exactly what we like
%    to achieve.
% \changes{v1.5?}{1994/?/?}{Do splitting to zero here}
%    \begin{macrocode}
   \setbox\@tempboxa\vsplit\mult@box to\z@
%    \end{macrocode}
%    Next we try to find a suitable starting point for the calculation
%    of the column height.  It should be less than the height finally
%    chosen, but large enough to reach this final value in only a few
%    iterations.  The formula which is now implemented will try to
%    start with the nearest value which is a multiple of
%    |\baselineskip|. The coding is slightly tricky in \TeX{} and
%    there are perhaps better ways \ldots
% \changes{v1.4d}{1992/03/04}{New algorithm for start height}
%    \begin{macrocode}
   \@tempdima\ht\mult@box
   \advance\@tempdima\dp\mult@box
   \divide\@tempdima\col@number
%    \end{macrocode}
%    The code above sets |\@tempdima| to the length of a column if we
%    simply divide the whole box into equal pieces. To get to the next
%    lower multiple of |\baselineskip| we convert this dimen to a
%    number (the number of scaled points) then divide this by
%    |\baselineskip| (also in scaled points) and then multiply this
%    result with |\baselineskip| assigning the result to |\dimen@|.
%    This makes |\dimen@| $\leq$ to |\@tempdimena|.
%    \begin{macrocode}
   \count@\@tempdima
   \divide\count@\baselineskip
   \dimen@\count@\baselineskip
%    \end{macrocode}
%    Next step is to correct our result by taking into account the
%    difference between |\topskip| and |\baselineskip|. We start by
%    adding |\topskip|; if this makes the result too large then we
%    have to subtract one |\baselineskip|.
%    \begin{macrocode}
   \advance\dimen@\topskip
   \ifdim \dimen@ >\@tempdima
     \advance\dimen@-\baselineskip
   \fi
%    \end{macrocode}
%    At the user's request we start with a higher value (or lower, but
%    this usually only increases the number of tries).
%    \begin{macrocode}
   \advance\dimen@\c@unbalance\baselineskip
%    \end{macrocode}
%    We type out statistics if we were asked to do so.
% \changes{v1.4f}{1992/04/28}{\cs{on@line} added to tracing info}
%    \begin{macrocode}
   \mult@info\@ne
      {Balance columns\on@line:
        \ifnum\c@unbalance=\z@\else
       (off balance=\number\c@unbalance)\fi
      \@gobbletwo}%
%    \end{macrocode}
%    But we don't allow nonsense values for a start.
%    \begin{macrocode}
   \ifnum\dimen@<\topskip
     \mult@info\@ne
       {Start value
          \the\dimen@  \space ->
          \the\topskip \space (corrected)}%
     \dimen@\topskip
   \fi
%    \end{macrocode}
%    Now we try to find the final column height.  We start by setting
%    |\vbadness| to infinity (i.e.\ $10000$) to suppress
%    underfull box reports while we are trying to find an acceptable
%    solution.  We do not need to do it in a group since at the end of
%    the output routine everything will be restored. The setting of
%    the final columns will nearly always produce underfull boxes with
%    badness $10000$ so there is no point in warning the user about
%    it.
% \changes{v1.2a}{1990/02/05}{Group around main loop removed.}
%    \begin{macrocode}
   \vbadness\@M
%    \end{macrocode}
%    We also allow for overfull boxes while we trying to split the
%    columns. They can easily happen if we have objects with unusual depth. 
% \changes{v1.8h}{2014/09/12}{Do not report overfull}
%    \begin{macrocode}
   \vfuzz \maxdimen
%    \end{macrocode}
%    The variable |\last@try| will hold the dimension used in the
%    previous trial splitting. We initialize it with a negative value.
% \changes{v1.5?}{1994/?/?}{Initialize \cs{last@try}}
%    \begin{macrocode}
   \last@try-\p@
   \loop
%    \end{macrocode}
%    In order not to clutter up \TeX{}'s valuable main memory with
%    things that are no longer needed, we empty all globally used box
%    registers. This is necessary if we return to this point after an
%    unsuccessful trial.  We use |\process@cols| for this purpose,
%    starting with |\mult@grightbox|.  Note the extra braces around
%    this macro call.  They are needed since \PlainTeX{}'s
%    |\loop|\ldots{}\allowbreak|\repeat| mechanism cannot be nested on
%    the same level of grouping.
% \changes{v1.5a}{1992/11/04}{New box mechanism}
%    \begin{macrocode}
    {\process@cols\mult@grightbox
           {\global\setbox\count@
                   \box\voidb@x}}%
%    \end{macrocode}
%    The contents of box |\mult@box| are now copied globally to
%    box~|\mult@grightbox|.  (This will be the right-most column, as
%    we shall see later.)
%    \begin{macrocode}
    \global\setbox\mult@grightbox
           \copy\mult@box
%    \end{macrocode}
%    We start with the assumption that the trial will be successful.
%    If we end up with a solution that is too bad we set
%    |too@bad| to \texttt{true}. We also assume that all forced breaks
%    (if any) will be used during balancing. If this is not the case
%    we record this in |forcedbreak@leftover|.
% \changes{v1.5b}{1992/11/05}{New badness mechanism}
% \changes{v1.8k}{2015/03/21}{Init \cs{ifforcedbreak@leftover}}
%    \begin{macrocode}
%<*badness>
   \too@badfalse
   \forcedbreak@leftoverfalse
%</badness>
%    \end{macrocode}
%    Using |\vsplit| we extract the other columns from box register
%    |\mult@grightbox|.  This leaves box register |\mult@box|
%    untouched so that we can start over again if this trial was
%    unsuccessful.
%    \begin{macrocode}
   {\process@cols\mult@firstbox{%
         \global\setbox\count@
         \vsplit\mult@grightbox to\dimen@
%    \end{macrocode}
%    After splitting we need to ensure that there isn't any space at
%    the bottom, so we rebox once more.
% \changes{v1.8b}{2013/08/28}{Remove discardable items at the
%                             end of split boxes}
% \changes{v1.8k}{2015/03/21}{Do not use \cs{remove@discardable@items}
%    here}
%    \begin{macrocode}
         \global\setbox\count@
                 \vbox to\dimen@
                  {\unvbox\count@}%
%    \end{macrocode}
%    After every split we check the badness of the resulting column,
%    normally the amount of extra white in the column.
%    \begin{macrocode}
%<*badness>
         \ifnum\c@tracingmulticols>\@ne
           \@tempcnta\count@
           \advance\@tempcnta-\mult@grightbox
           \divide\@tempcnta \tw@
           \message{^^JColumn
              \number\@tempcnta\space
               badness: \the\badness\space}%
         \fi
%    \end{macrocode}
%    If this badness is larger than the allowed column badness
%    we reject this solution by setting |too@bad| to \texttt{true}.
%    \begin{macrocode}
         \ifnum\badness>\c@columnbadness
           \ifnum\c@tracingmulticols>\@ne
             \message{too bad
                  (>\the\c@columnbadness)}%
           \fi
           \too@badtrue
         \fi
%</badness>
                        }}%
%    \end{macrocode}
%    There is one subtle point here: while all other constructed boxes
%    have a depth that is determined by |\splitmaxdepth| and/or
%   |\boxmaxdepth| the last box
%    will get a natural depth disregarding the original setting and
%    the value of |\splitmaxdepth| or |\boxmaxdepth|. This means that
%    we may end up with a very large depth in box |\mult@grightbox|
%    which would
%    make the result of the testing incorrect. So we change the value
%    by unboxing the box into itself.
%    \begin{macrocode}
   \global\setbox\mult@grightbox
      \vbox{\unvbox\mult@grightbox}%
%    \end{macrocode}
%    We also save a copy |\mult@firstbox| at its ``natural'' size
%    for later use.
%    \begin{macrocode}
   \setbox\mult@nat@firstbox
      \vbox{\unvcopy\mult@firstbox}%
%    \end{macrocode}
%    After |\process@cols| has done its job we have the following
%    situation:
%    \begin{center}
%      \begin{tabular}{r@{$\:\:\longleftarrow\:\:$}l}
%        box |\mult@rightbox| & all material \\
%        box |\mult@gfirstbox| & first column \\
%        box |\mult@gfirstbox|${}+2$ & second column \\
%        \multicolumn{1}{c}{$\vdots$} &
%        \multicolumn{1}{c}{$\vdots$} \\
%        box |\mult@grightbox| & last column
%      \end{tabular}
%    \end{center}
%    We report the height of the first column, in brackets
%    the natural size is given.
% \changes{v1.5?}{1994/?/?}{Show natural size}
%    \begin{macrocode}
    \ifnum\c@tracingmulticols>\@ne
       \message{^^JFirst column
           = \the\dimen@\space
           (\the\ht\mult@nat@firstbox)}\fi
%    \end{macrocode}
%    If |\raggedcolumns| is in force older releases of this file also
%    shrank the first column to its natural height at this point.
%    This was done so that the first column doesn't run short compared
%    to later columns but it is actually producing incorrect results
%    (overprinting of text) in boundary cases, so since version v1.5q
%    |\raggedcolumns| means allows for all columns to run slightly short.
% \changes{v1.5q}{1998/01/19}{Do not reset \cs{mult@firstbox} (pr2739)}
%    \begin{macrocode}
%    \ifshr@nking
%      \global\setbox\mult@firstbox
%             \copy\mult@nat@firstbox
%    \fi
%    \end{macrocode}
%    Then we give information about the last column.\footnote{With
%    \TeX{} version 3.141 it is now possible to use \LaTeX's
%    \cs{newlinechar} in the \cs{message} command, but
%    people with older \TeX{} versions will now get
%    \texttt{\string^\string^J} instead of a new line on the screen.}
% \changes{v1.4a}{1992/02/12}{Changed to proper \cs{endlinechar} in\cs{message}}
%    \begin{macrocode}
    \ifnum\c@tracingmulticols>\@ne
      \message{<> last column =
               \the\ht\mult@grightbox^^J}%
%    \end{macrocode}
%    Some tracing code that we don't compile into the production version
%    unless asked for. It will produce huge listings of the boxes
%    involved in balancing in the transcript file.
% \changes{v1.6f}{2004/07/03}{need to use \cs{mult@grightbox} in the loop}
%    \begin{macrocode}
%<*debug>
      \ifnum\c@tracingmulticols>4
         {\showoutput
          \batchmode
          \process@cols\mult@grightbox
           {\showbox\count@}}%
          \errorstopmode
      \fi
%</debug>
     \fi
%    \end{macrocode}
%    We check whether our trial was successful.  The test used is very
%    simple: we merely compare the first and the last column.  Thus
%    the intermediate columns may be longer than the first if
%    |\raggedcolumns| is used.  If the right-most column is
%    longer than the first then we start over with a larger value for
%    |\dimen@|.
% \changes{v1.3c}{1991/03/03}{\cs{global}\cs{advance} left over from older code}
%    \begin{macrocode}
    \ifdim\ht\mult@grightbox >\dimen@
%    \end{macrocode}
%    If the height of the last box is too large we mark this trial as
%    unsuccessful.
% \changes{v1.5v}{1999/07/18}{Added tracing statements for
%                             trial unsuccessful}
%    \begin{macrocode}
%<*badness>
      \too@badtrue
      \ifnum\c@tracingmulticols>\@ne
         \typeout{Rejected: last
                 column too large!}%
      \fi
    \else
%    \end{macrocode}
% \changes{v1.5v}{1999/07/18}{Check last column if it contains forced
%    break and reject trial if that is the case}
%
%    To ensure that there isn't a forced break in the last column we
%    try to split off a box of size |\maxdimen| from |\mult@grightbox|
%    (or rather from a copy of it). This should result in a void box
%    after the split, unless there was a forced break somewhere within
%    the column in which case the material after the break would have
%    stayed in the box.
%    \begin{macrocode}
      \setbox\@tempboxa
           \copy\mult@grightbox
      \setbox\z@\vsplit\@tempboxa to\maxdimen
      \ifvoid\@tempboxa
%    \end{macrocode}
%    Thus if |\@tempboxa| is void we  have a valid solution.
%    In this case we take a closer
%    look at the last column to decide if this column should be made
%    as long as all other columns or if it should be allowed to be
%    shorter.
%    For this we first have to rebox the column into a box of the
%    appropriate height. If tracing is enabled we then display the
%    badness for this box.
%    \begin{macrocode}
        \global\setbox\mult@grightbox
           \vbox to\dimen@
              {\unvbox\mult@grightbox}%
        \ifnum\c@tracingmulticols>\@ne
          \message{Final badness:
                   \the\badness}%
        \fi
%    \end{macrocode}
%    We then compare this badness with the allowed badness for the final
%    column. If it does not exceed this value we use the box, otherwise
%    we rebox it once more and add some glue at the bottom.
% \changes{v1.8h}{2014/09/12}{Use \cs{vfilmaxdepth}}
% \changes{v1.8j}{2015/03/07}{Use \cs{vfil} in this case}
%    \begin{macrocode}
        \ifnum\badness>\c@finalcolumnbadness
          \global\setbox\mult@grightbox
           \vbox to\dimen@
              {\unvbox\mult@grightbox\vfil}%
           \ifnum\c@tracingmulticols>\@ne
             \message{ setting natural
              (> \the\c@finalcolumnbadness)}%
           \fi
        \fi
%    \end{macrocode}
%    If |\@tempboxa| above was not void our trial was unsuccessful and
%    we report this fact and try again.
% \changes{v1.6f}{2004/07/03}{\texttt{/colbreak} guard in the wrong position}
%    \begin{macrocode}
      \else
%    \end{macrocode}
%    If we have unprocessed forced breaks we normally reiterate with a
%    larger column size to fit them in eventually. However, if there
%    are simply too many of them (e.g., 3 forced breaks but only 2
%    columns to balance) then this will never succeed and and we would
%    continue growing the columns until we hit the largest possible
%    column size. So in addition we check how big the column size is
%    compared to available room and if we exceed this by
%    |\maxbalancingoverflow| we give up and instead of balancing cut
%    another normal page. To be indicate this case we set
%    |forcedbreak@leftover| to true.
% \changes{v1.8k}{2015/03/21}{Watch out for columns growing too far in
%    case of forced breaks}
% \changes{v1.8l}{2015/03/25}{Added additional tracing if column overflows}
%    \begin{macrocode}
       \@tempdima\@colroom
       \advance\@tempdima \maxbalancingoverflow
       \ifdim \dimen@ < \@tempdima
        \too@badtrue
         \ifnum\c@tracingmulticols>\@ne
           \typeout{Rejected: unprocessed
             forced break(s) in last column!}%
         \fi
       \else
         \forcedbreak@leftovertrue
         \ifnum\c@tracingmulticols>\@ne
          \typeout{Failed: columns too large
            with unprocessed forced break(s)!}%
         \fi
       \fi
      \fi
    \fi
%    \end{macrocode}
%    If the natural height of the first box is smaller than the
%    current trial size but is larger than the previous trial size it
%    is likely that we have missed a potentially better
%    solution. (This could have happened if for some reason our first
%    trial size was too high.) In that case we dismiss this trial and
%    restart using the natural height for the next trial.
%    \begin{macrocode}
    \ifdim\ht\mult@nat@firstbox<\dimen@
      \ifdim\ht\mult@nat@firstbox>\last@try
        \too@badtrue
        \ifnum\c@tracingmulticols>\@ne
           \typeout{Retry: using natural
                    height of first column!}%
        \fi
        \dimen@\ht\mult@nat@firstbox
        \last@try\dimen@
        \advance\dimen@-\p@
      \fi
    \fi
%    \end{macrocode}
%    Finally the switch |too@bad| is tested. If it was made true
%    either earlier on or due to a rightmost column being too large
%    we try again with a slightly larger value for |\dimen@|.
%    \begin{macrocode}
    \iftoo@bad
%</badness>
      \advance\dimen@\p@
    \repeat
%    \end{macrocode} 
%    If we come out of the loop with the switch |forcedbreak@leftover|
%    set to true then balancing has failed and we should cut a normal
%    page. We indicate this below with |\too@badtrue| when any of the
%    columns get too high, so we set this flag here too in order to
%    get the same processing logic.\footnote{Should get cleaned up as
%    we now have two different routes to reach this part of the
%    processing.}
% \changes{v1.8k}{2015/03/21}{}
%    \begin{macrocode}
    \ifforcedbreak@leftover
       \too@badtrue
    \else
%    \end{macrocode}
%    At that point |\dimen@| holds the height that was determined by
%    the balancing loop.
%    If that height for the columns turns out to be larger
%    than the available space (which is |\@colroom|) we squeeze the
%    columns into the space assuming that they will have enough
%    shrinkability to allow this.\footnote{This might be wrong, since
%    the shrinkability that accounts for the amount of material might
%    be present only in some columns. But it is better to try then to
%    give up directly.}
%    However, this squeezing should only be done if we are balancing
%    columns on the main galley and \emph{not} if we are building a
%    boxed multicol (in the latter case the current |\@colroom| is
%    irrelevant since the produced box might be moved anywhere at a
%    later stage).
% \changes{v1.3c}{1991/03/03}{Limit column height to \cs{@colroom}}
% \changes{v1.5q}{1998/01/19}{Removed setting \cs{dimen@} (pr2739)}
% \changes{v1.5y}{2000/06/10}{Limit column height only in unrestricted
%    mode (pr/3212)}
%    \begin{macrocode}
     \if@boxedmulticols\else
       \ifdim\dimen@>\@colroom
         \dimen@\@colroom
       \fi
     \fi
%    \end{macrocode}
%    Then we move the contents of the odd-numbered box registers to
%    the even-numbered ones, shrinking them if requested.
%    We have to use |\vbox| not |\vtop| (as it was done in
%    the first versions) since otherwise the resulting boxes will have
%    no height (\TB\/ page 81). This would mean that extra
%    |\topskip|  is added when the boxes are returned to the
%    page-builder via |\page@sofar|.
% \changes{v1.3a}{1990/05/20}{Changed \cs{vtop} to \cs{vbox}.}
%    \begin{macrocode}
     \process@cols\mult@rightbox
         {\@tempcnta\count@
          \advance\@tempcnta\@ne
%    \end{macrocode}
%    when putting the final column together we want overfull
%    information:
%    \begin{macrocode}
          \vfuzz\z@
          \setbox\count@\vbox to\dimen@
             {%
%    \end{macrocode}
%
% \changes{v1.8h}{2014/09/12}{Use \cs{vfilmaxdepth}}
%    \begin{macrocode}
              \vskip \z@
                \@plus-\multicolundershoot
                \@minus-\multicolovershoot
              \unvbox\@tempcnta
              \ifshr@nking\vfilmaxdepth\fi
             }%
%    \end{macrocode}
%    If the resulting box is overfull there was too much material to
%    fit into the available space. The question though is how much? If
%    it wasn't more than |\maxbalancingoverflow| we accept it still to
%    avoid getting very little material for the next page (which we
%    would then have difficulties to balance).
% \changes{v1.8a}{2011/12/20}{Balancing concept improved}
%    \begin{macrocode}
          \ifnum\badness>\@M
            \vfuzz\maxdimen % no overfull warning
            \setbox\@tempboxa \vbox to\dimen@
                 {\vskip-\maxbalancingoverflow
                  \unvcopy\count@}%
            \ifnum\badness>\@M
              \mult@info\@ne
                {Balanced column more than
                  \the\maxbalancingoverflow\space
                  too large}%
%    \end{macrocode}
%    Fail the balancing attempt:
%    \begin{macrocode}
              \too@badtrue
            \else
%    \end{macrocode}
%    Otherwise report that there is a problem but within the accepted
%    boundary.
%    \begin{macrocode}
              \mult@info\@ne
                {Balanced column
                 too large, but less than
                 \the\maxbalancingoverflow}%
            \fi
          \fi
         }%
%    \end{macrocode}
%    Finally end the |\ifforcedbreak@leftover| conditional.
% \changes{v1.8k}{2015/03/21}{Finish the new conditional}
%    \begin{macrocode}
   \fi
}
%    \end{macrocode}
%  \end{macro}
%
%
% \begin{macro}{\maxbalancingoverflow}
%    Amount that balancing is allowed to overflow the available column
%    space. We default to 12pt which means about one line in most
%    layouts.
% \changes{v1.8a}{2011/12/20}{\cs{maxbalancingoverflow} parameter added}
%    \begin{macrocode}
\newdimen\maxbalancingoverflow
\maxbalancingoverflow=12pt
%    \end{macrocode}
%  \end{macro}

% \end{multicols}
%
% \begin{multicols}{2}[\subsection{The box allocations}]
%
% \begin{macro}{\mult@rightbox}
% \begin{macro}{\mult@grightbox}
% \begin{macro}{\mult@firstbox}
% \begin{macro}{\mult@gfirstbox}
%    Early releases of these macros used the first box registers
%    0, 2, 4,\ldots\ for global boxes and 1, 3, 5,\ldots\ for the
%    corresponding local boxes. (You might still find some traces
%    of this setup in the documentation, sigh.) This produced a problem
%    at the moment we had more than 5 columns because then officially
%    allocated boxes were overwritten by the algorithm.
%    The new release now uses private box registers
%    \begin{macrocode}
\newbox\mult@rightbox
\newbox\mult@grightbox
\newbox\mult@gfirstbox
\newbox\mult@firstbox
\newbox\@tempa\newbox\@tempa
\newbox\@tempa\newbox\@tempa
\newbox\@tempa\newbox\@tempa
\newbox\@tempa\newbox\@tempa
\newbox\@tempa\newbox\@tempa
\newbox\@tempa\newbox\@tempa
\newbox\@tempa\newbox\@tempa
\newbox\@tempa\newbox\@tempa
\newbox\@tempa
\let\@tempa\relax
%    \end{macrocode}
% \end{macro}
% \end{macro}
% \end{macro}
% \end{macro}
%
%
% \end{multicols}
%
%
% \begin{multicols}{2}[\section{New macros and hacks for version 1.2}]
%
% \begin{macro}{\emergencystretch}
% \begin{macro}{\setemergencystretch}
%    If we don't use \TeX{} 3.0 |\emergencystretch| is undefined
%    so in this case we simply add it as an unused \meta{dimen}
%    register.
% \changes{v1.4j}{1992/06/25}{Setting of \cs{emergencystretch} on top
%                  removed.}
%    \begin{macrocode}
\@ifundefined{emergencystretch}
     {\newdimen\emergencystretch}{}
%    \end{macrocode}
% \changes{v1.2a}{1990/02/05}{Macro added.}
%    My tests showed that the following formula worked pretty well.
%    Nevertheless the |\setemergencystretch| macro also gets
%    |\hsize| as second argument to enable the user to try
%    different formulae.
%    \begin{macrocode}
\def\setemergencystretch#1#2{%
   \emergencystretch 4pt
   \multiply\emergencystretch#1}
%    \end{macrocode}
% \end{macro}
% \end{macro}


% \begin{macro}{\set@floatcmds}
% \changes{v1.2a}{1990/02/05}{Macro added.}
% \changes{v1.5g}{1994/06/07}{Updated since floats have changed}
% \changes{v1.5j}{1994/06/07}{Updated since floats have changed again}
% \changes{v1.5l}{1995/10/19}{Added \cs{@largefloatcheck}}
% \changes{v1.6g}{2006/02/23}{Added \cs{@minipagefalse}}
% \changes{v1.6h}{2008/12/05}{Use \cs{@endfloatbox} to better support
%   the modifications done by the float package}
%    Even if this should be used as a hook we use a |@| in the
%    name since it is more for experts.
%    \begin{macrocode}
\def\set@floatcmds{%
 \let\@dblfloat\@dbflt
 \def\end@dblfloat{\@endfloatbox
   \@largefloatcheck
   \outer@nobreak
%    \end{macrocode}
%    This is cheap (deferring the floats until after the current page)
%    but any other solution would go deep into \LaTeX's output
%    routine and I don't like to work on it until I know which parts
%    of the output routine have to be reimplemented anyway for
%    \LaTeX3.
%    \begin{macrocode}
   \ifnum\@floatpenalty<\z@
%    \end{macrocode}
%    We have to add the float to the |\@deferlist| because we assume
%    that outside the \mc{} environment we are in one column mode.
%    This is not entirely correct, I already used the \mc{}
%    environment inside of \LaTeX{}s |\twocolumn| declaration but it
%    will do for most applications.
%    \begin{macrocode}
     \@cons\@deferlist\@currbox
   \fi
   \ifnum\@floatpenalty=-\@Mii
     \@Esphack
   \fi}}
%    \end{macrocode}
% \end{macro}
%
% \end{multicols}
%
% \begin{multicols}{2}[\subsection{Maintaining the mark registers}]
% \label{sec:v14}
%
% This section contains the routines that set the marks so that they
% will be handled correctly. They have been introduced with version~1.4.
%
%  \begin{macro}{\kept@topmark}
% \changes{v1.4h}{1992/06/04}{Init to double brace pair}
%  \begin{macro}{\kept@firstmark}
%  \begin{macro}{\kept@botmark}
%    First thing we do is to reserve three macro names to hold the
%    replacement text for \TeX's primitives |\firstmark|, |\botmark| and
%    |\topmark|. We initialize the first two to be empty and
%    |\kept@topmark| to contain  two empty pair of braces. This is
%    necessary since |\kept@topmark| is supposed to contain the last
%    mark from a preceding page and in \LaTeX{} any ``real'' mark must
%    contain two parts representing left and right mark information.
%    \begin{macrocode}
\def\kept@topmark{{}{}}
\let\kept@firstmark\@empty
\let\kept@botmark\@empty
%    \end{macrocode}
%  \end{macro}
%  \end{macro}
%  \end{macro}
%
%
%  \begin{macro}{\return@nonemptymark}
%    Sometimes we want to return the value of a ``kept'' mark into a
%    |\mark| node on the main vertical list. This is done by the
%    function |\return@nonemptymark|. As the name suggests it only acts
%    if the replacement text of the kept mark is non-empty. This is done
%    to avoid adding an empty mark when no mark was actually present. If
%    we would nevertheless add such a mark it would be regarded as a
%    valid |\firstmark| later on.
%    \begin{macrocode}
\def\return@nonemptymark#1#2{%
  \ifx#2\@empty
  \else
%    \end{macrocode}
%    For debugging purposes we take a look at the value of the kept mark
%    that we are about to return. This code will get stripped out for
%    production.
%    \begin{macrocode}
%<*marktrace>
    \mult@info\tw@
      {Returned #1 mark:\MessageBreak
       \meaning#2}%
%      \nobreak
%    \fi
%</marktrace>
%    \end{macrocode}
%    Since the contents of the mark may be arbitrary \LaTeX{} code we
%    better make sure that it doesn't get expanded any further. (Some
%    expansion have been done already during the execution of
%    |\markright| or |\markboth|.) We therefore use the usual mechanism
%    of a toks register to prohibit expansion.\footnote{Due to the
%    current definition of \cs{markright} etc.\ it wouldn't
%    help to define the \cs{protect} command to prohibit
%    expansion as any \cs{protect} has already vanished due to
%    earlier expansions.}
% \changes{v1.4n}{1992/09/10}{Make marks robust}
% \changes{v1.5t}{1999/03/22}{re-add \cs{mark} command which was commented out
%       by mistake at some point in 1998 (pr/2978)}
%    \begin{macrocode}
    \toks@\expandafter{#2}%
    \mark{\the\toks@}%
%    \end{macrocode}
%    We don't want any breakpoint between such a returned mark and the
%    following material (which is usually just the box where the mark
%    came from).
%    \begin{macrocode}
    \nobreak
  \fi}
%    \end{macrocode}
%  \end{macro}
%
%
%  \begin{macro}{\get@keptmarks}
%    If we have some material in a box register we may want to get the
%    first and the last mark out of this box. This can be done with
%    |\get@keptmarks| which takes one argument: the box register number
%    or its nick name defined by |\newbox|.
%    \begin{macrocode}
\def\get@keptmarks#1{%
%    \end{macrocode}
%    For debugging purposes we take a look at the current dimensions
%    of the box since in earlier versions of the code I made some
%    mistakes in this area.
%    \begin{macrocode}
%<*debug>
      \typeout{Mark box #1 before:
               ht \the\ht#1, dp \the\dp#1}%
%</debug>
%    \end{macrocode}
%    Now we open a new group an locally copy the box to itself. As a
%    result any operation, i.e.\ |\vsplit|, will only have a local
%    effect. Without this trick the box content would get lost up to
%    the level where the last assignment to the box register was done.
%    \begin{macrocode}
  \begingroup
   \vbadness\@M
   \setbox#1\copy#1%
%    \end{macrocode}
%    Now we split the box to the maximal possible dimension. This
%    should split off the full contents of the box so that effectively
%    everything is split off. As a result |\splitfirstmark| and
%    |\splitbotmark| will contain the first and last mark in the box
%    respectively.
%    \begin{macrocode}
   \setbox#1\vsplit#1to\maxdimen
%    \end{macrocode}
%    Therefore we can now set the kept marks which is a global
%    operation and afterwards close the group. This will restore the
%    original box contents.
%    \begin{macrocode}
   \set@keptmarks
 \endgroup
%    \end{macrocode}
%    For debugging we take again a look at the box dimension which
%    shouldn't have changed.
%    \begin{macrocode}
%<*debug>
    \typeout{Mark box #1 \space after:
             ht \the\ht#1, dp \the\dp#1}%
%</debug>
}
%    \end{macrocode}
%  \end{macro}
%
%
%  \begin{macro}{\set@keptmarks}
%    The macro |\set@keptmarks| is responsible for setting
%    |\kept@firstmark| and |\kept@botmark|, by checking the current
%    values for |\splitfirstmark| and |\splitbotmark|.
%    \begin{macrocode}
\def\set@keptmarks{%
%    \end{macrocode}
%    If |\kept@firstmark| is empty we assume that it isn't set. This
%    is strictly speaking not correct as we loose the ability to have
%    marks that are explicitly empty, but for standard \LaTeX{}
%    application it is sufficient. If it is non-empty we don't change
%    the value---within the output routines it will then be restored
%    to |\@empty|.
%    \begin{macrocode}
   \ifx\kept@firstmark\@empty
%    \end{macrocode}
%    We now put the contents of |\splitfirstmark| into
%    |\kept@firstmark|. In the case that there wasn't any mark at all
%    |\kept@firstmark| will not change by that operation.
%    \begin{macrocode}
     \expandafter\gdef\expandafter
        \kept@firstmark
        \expandafter{\splitfirstmark}%
%    \end{macrocode}
%    When debugging we show the assignment but only when something
%    actually happened.
%    \begin{macrocode}
%<*marktrace>
     \ifx\kept@firstmark\@empty\else
       \mult@info\tw@
         {Set kept first mark:\MessageBreak
          \meaning\kept@firstmark%
          \@gobbletwo}%
     \fi
%</marktrace>
   \fi
%    \end{macrocode}
%    We always try to set the bottom mark to the |\splitbotmark| but
%    of course only when there has been a |\splitbotmark| at all.
%    Again, we assume that an empty |\splitbotmark| means that the
%    split off box part didn't contain any marks at all.
%    \begin{macrocode}
   \expandafter\def\expandafter\@tempa
      \expandafter{\splitbotmark}%
   \ifx\@tempa\@empty\else
      \global\let\kept@botmark\@tempa
%<*marktrace>
      \mult@info\tw@
        {Set kept bot mark:\MessageBreak
         \meaning\kept@botmark%
         \@gobbletwo}%
%</marktrace>
   \fi}%
%    \end{macrocode}
%  \end{macro}
%
%
%  \begin{macro}{\prep@keptmarks}
%    The |\prep@keptmarks| function is used to initialize the kept
%    marks from the contents of  |\partial@page|, i.e.\ the box that
%    holds everything from the top of the current page prior to
%    starting the \mc{} environment. However, such a box is only
%    available if we are not producing a boxed \mc{}.
%    \begin{macrocode}
\def\prep@keptmarks{%
   \if@boxedmulticols \else
     \get@keptmarks\partial@page
   \fi}
%    \end{macrocode}
%  \end{macro}
%
%
%  \begin{macro}{\remove@discardable@items}
%    There are situations when we may have some space at the end of a
%    column and this macro here will attempt to get rid of it. The
%    typical \LaTeX{} sequence is a series of selfcanceling glues so
%    if we remove them recursively we are usually fine.
%
%    Special care is needed with handling |\vspace*| as that
%    corresponds to |\penalty10000|, |\vskip <skip>|, followed by
%    |\vskip 0pt|. If we see this sequence going backwards in the the
%    vertical list we assume that this is a ``desired'' space. We
%    therefore stop the recursion and reinsert the spaces.
%
%    As the multicol code sometimes add an explicit penalty at the end
%    of a column we first attempt to remove it in case it is there.
% \changes{v1.8m}{2015/03/31}{Another rewrite of
%     \cs{remove@discardable@items} hopefully okay now}
%    \begin{macrocode}
\skip0=0pt
\edef\the@zero@skip{\the\skip0}
\def\remove@discardable@items{%
     \unpenalty
%    \end{macrocode}
%    Save a previous skip (if there) and then remove it, we can't
%    really tell the difference between no skip an a skip of zero but
%    that's life.
%    \begin{macrocode}
     \edef\@tempa{\the\lastskip}%
%\typeout{s1=\@tempa}%
     \unskip
%    \end{macrocode}
%    If it was a zero skip (or none) we save the next previous skip
%    (if any).
%    \begin{macrocode}
     \ifx\@tempa\the@zero@skip
       \edef\@tempb{\the\lastskip}%
%\typeout{s2=\@tempb}%
%    \end{macrocode}
%    If this one again was zero (or more likely not there in the first
%    place) we stop.
%    \begin{macrocode}
       \ifx\@tempb\the@zero@skip
       \else
%    \end{macrocode}
%    Otherwise we remove this ``real'' skip. Then we look if it was
%    preceeded by a penalty of 10000 (i.e., a |\nobreak|)
%    \begin{macrocode}
         \unskip
%\typeout{p=\lastpenalty}%
         \ifnum \lastpenalty=\@M
%    \end{macrocode}
%    If so this was a |\vspace*| or something equivalent to
%    it. Therefore we reintroduce the skips and stop. Otherwise we
%    recurse.
%    \begin{macrocode}
           \vskip\@tempb\vskip\@tempa\relax
         \else 
           \remove@discardable@items
         \fi
       \fi
     \else
%    \end{macrocode}
%    If the first skip was a non-zero skip we recurse as well.
%    \begin{macrocode}
       \remove@discardable@items
     \fi
}
%    \end{macrocode}
%  \end{macro}
%
%    \begin{macrocode}
%<*badness>
\newif\iftoo@bad
\def\too@badtrue{\global\let\iftoo@bad\iftrue}
\def\too@badfalse{\global\let\iftoo@bad\iffalse}
%    \end{macrocode}
%    
% \changes{v1.8k}{2015/03/21}{The new switch}
%    \begin{macrocode}
\newif\ifforcedbreak@leftover
%    \end{macrocode}
%
% \begin{macro}{\c@columnbadness}
% \begin{macro}{\c@finalcolumnbadness}
%    \begin{macrocode}
\newcount\c@columnbadness
\c@columnbadness=10000
\newcount\c@finalcolumnbadness
\c@finalcolumnbadness=9999

\newdimen\last@try

%    \end{macrocode}
%
% \changes{v1.5z1}{2003/02/17}{Change wrong default for
%    \cs{multicolovershoot} to zero (pr/3465).}
%    \begin{macrocode}
\newdimen\multicolovershoot
\newdimen\multicolundershoot
\multicolovershoot=0pt
\multicolundershoot=2pt
\newbox\mult@nat@firstbox
%</badness>
%    \end{macrocode}
% \end{macro}
% \end{macro}
%
%  \begin{macro}{\mult@info}
%    A helper for producing info messages
%    \begin{macrocode}
\def\mult@info#1#2{%
  \ifnum\c@tracingmulticols>#1%
   \GenericWarning
       {(multicol)\@spaces\@spaces}%
       {Package multicol: #2}%
  \fi
}
%    \end{macrocode}
%  \end{macro}
%
% \end{multicols}
%
%
% \begin{multicols}{2}[\section{Fixing the
%                        \cs{columnwidth}}]
%
%  \begin{macro}{\@footnotetext}
% \changes{v1.5o}{1997/11/16}{Redefinition added pr/2664.}
% \changes{v1.5r}{1998/08/17}{Use \cs{@footnotetext} but with
%                             local change to \cs{columnwidth}.}
%  \begin{macro}{\mult@footnotetext}
% \changes{v1.5r}{1998/08/17}{Macro removed again.}
%    If we store the current column width in |\columnwidth| we have
%    to redefine the internal |\@footnotetext| macro to use |\textwidth|
%    for the width of the footnotes rather than using the original
%    definition.
%
%    Starting with version v1.5r this is now done in a way that the original
%    definition is still used, except that locally |\columnwidth| is set to
%    |\textwidth|.
%
%    This solves two problems: first redefinitions of |\@footnotetext|
%    done by a class will correctly survive and second if multicols is
%    used inside a minipage environment the special definition of
%    |\@footnotetext| in that environment will be picked up and not the
%    one for the main galley (the latter would result in all footnotes
%    getting lost in that case).
%
%    See the definition of the |\multicols| command further up for the exact
%    code.
%  \end{macro}
%  \end{macro}
% \end{multicols}
%
%
% \begin{multicols*}{2}[\section{Further extensions}]
%
% This section does contain code for extensions added to this package
% over time. Not all of them may be active, some might sit dormant and
% wait for being activated in some later release.
%
% \subsection{Not balancing the columns}
%
%  This is fairly trivial to implement. we just have to disable the
%  balancing output routine and replace it by the one that ships out
%  the other pages.
%
%  \begin{macro}{\multicols*}
% \changes{v1.5q}{1998/01/19}{Macro added}
%  The code for this environment was suggested by Matthias Clasen.
%    \begin{macrocode}
%<*nobalance>
 \@namedef{multicols*}{%
%    \end{macrocode}
%    If we are not on the main galley, i.e., inside a box of some
%    sort, that approach will not work since we don't have a vertical
%    size for the box so we better warn that we balance anyway.
%    \begin{macrocode}
   \ifinner
     \PackageWarning{multicol}%
       {multicols* inside a box does
        not make sense.\MessageBreak
        Going to balance anyway}%
   \else
     \let\balance@columns@out
         \multi@column@out
   \fi
   \begin{multicols}
}
%    \end{macrocode}
%  \end{macro}
%
%  \begin{macro}{\endmulticols*}
%    When ending the environment we simply end the inner
%    \texttt{multicols} environment, except that we better also stick
%    in some stretchable vertical glue so that the last column still
%    containing text is not vertically stretched out. 
%
%    We do this as follows: first we ensure that we are back in vertical mode
%    and then we cancel out |\lastskip| if it was
%    positive (in case of a negative glue we assume that it was
%    deliberate, for a deliberate positive glue one needs to use
%    |\vspace*|). We can't simply use |\remove@discardable@items| here
%    as this only works inside boxes but we are here on the main
%    vertical list.
%
%    Then we back up by |\prevdepth| but not more than |\boxmaxdepth|
%    so that a baseline of the last box is now at the bottom. This way
%    the material will align properly in case something like |\vfill|
%    spreads it out after all.
%    Finally we append |\vfil| to put white space at the bottom of the
%    column, but we only do this if we aren't anyway doing |\raggedcolumns|.
% \changes{v1.5q}{1998/01/19}{Macro added}
% \changes{v1.8h}{2014/09/12}{Use \cs{vfilmaxdepth}}
% \changes{v1.8i}{2014/10/28}{Add \cs{null} to hide the final fill and only add
%                             vertical space if not doing \cs{raggedcolumns}}
% \changes{v1.8j}{2015/03/07}{Redesign the whole approach.}
% \changes{v1.8k}{2015/03/21}{And a bit more redesign because of the
%    change in \cs{remove@discardable@items}}
% \changes{v1.8o}{2016/02/08}{Ensure we are back in vmode before using
%    \cs{prevdepth} (pr/4448)}
%    \begin{macrocode}
\@namedef{endmulticols*}{%
   \par
   \ifdim\lastskip>\z@ \vskip-\lastskip \fi
   \ifdim \prevdepth>\z@
     \vskip-\ifdim\prevdepth>\boxmaxdepth 
                  \boxmaxdepth
            \else \prevdepth \fi
   \fi
   \ifshr@nking\else
     \vfil
   \fi
   \end{multicols}}
%</nobalance>
%    \end{macrocode}
%  \end{macro}
%
%
% \subsection{Manual column breaking} \label{sec:colbreak}
%
% The problem with manual page breaks within \mc{} is the fact that
% during collection of material for all columns a page-forcing penalty
% (i.e. -10000 or higher) would stop the collecting pass which is not
% quite what is desired. On the other hand, using a penalty like -9999
% would mean that there would be occasions where the |\vsplit|ing
% operations within \mc{} would ignore that penalty and still choose a
% different break point.
%
% For this reason the current implementation
% uses a completely different approach. In a nutshell it extends the \LaTeX{}
% output routine handling by introducing an additional penalty flag
% (i.e., a penalty which is forcing but higher than -10000 so that the
% output routine can look at this value and thus knows why it has been
% called).
%
% Inside the output routine we test for this value and if it appears
% we do two things: save the galley up to this point in a special box
% for later use and reduce the |\vsize| by the height of the material
% seen. This way the forcing penalty is now hidden in that box and we
% can restart the collection process for the remaining
% columns. (This is done in |\speci@ls| above.)
%
% In the output routines that do the |\vsplit|ting either for
% balancing or for a full page we simply combine box~255 with the
% saved box thus getting a single box for  splitting which now
% contains forcing breaks in the right positions.
%
%
%  \begin{macro}{\columnbreak}
% \changes{v1.5u}{1999/05/25}{Macro added}
%    |\columnbreak| is modelled after |\pagebreak| except that we
%    generate a penalty -10005.
%    \begin{macrocode}
\mathchardef\@Mv=10005
\def\columnbreak{%
%    \end{macrocode}
%    We have to ensure that it is only used within a \mc{}
%    environment since if that penalty would be seen by the unmodified
%    \LaTeX{} output routine strange things would happen.
%    \begin{macrocode}
 \ifnum\col@number<\tw@
  \PackageError{multicol}%
   {\noexpand\columnbreak outside multicols}%
   {This command can only be used within
    a multicols or multicols* environment.}%
 \else
  \ifvmode
    \penalty -\@Mv\relax
  \else
    \@bsphack
    \vadjust{\penalty -\@Mv\relax}%
    \@esphack
  \fi
 \fi}
%    \end{macrocode}
%  \end{macro}
%
%
%  \begin{macro}{\colbreak@box}
% \changes{v1.5u}{1999/05/25}{Macro added}
%    Need a box to collect the galley up to the column break.
%    \begin{macrocode}
\newbox\colbreak@box
%</package>
%    \end{macrocode}
%  \end{macro}
%
%
% \subsection{Supporting right-to-left languages} \label{sec:RL}
%
%
% \changes{v1.7a}{2010/10/24}{RL language support added}
% \changes{v1.7b}{2011/12/18}{RL language support fixed}
%
%  \begin{macro}{\LR@column@boxes}
%    |\LR@column@boxes| is called when we are assembling the columns for left
%    to right typesetting. When we start we are inside an |\hbox| of
%    full width.
%    Left to right typesetting is fairly easy, we basically output
%    each column box intermixed with vertical rules and proper
%    spacing. As this happens inside a box of a defined width the
%    rules and the columns automatically get into the right positions.
%    \begin{macrocode}
\def\LR@column@boxes{%
%    \end{macrocode}
%    We loop through the columns with |\process@cols|
%    \begin{macrocode}
     \process@cols\mult@gfirstbox{%
%    \end{macrocode}
%    If the depth of the current box is larger than the maximum found
%    so far in |\dimen2| we update that register for later use.
% \changes{v1.6a}{2003/03/15}{Preparing for adjusting \cs{prevdepth}}
%    \begin{macrocode}
       \ifdim\dp\count@>\dimen\tw@
         \global\dimen\tw@\dp\count@ \fi
%    \end{macrocode}
%    If the \texttt{colaction} option is given we write out status
%    information about the current column, otherwise the next command
%    does nothing.
% \changes{v1.8e}{2014/04/23}{Support \cs{docolaction}}
%    \begin{macrocode}
       \mc@col@status@write
%    \end{macrocode}
%    The typeset box followed by the column rule material
%    \begin{macrocode}
       \box\count@
       \hss{\columnseprulecolor\vrule
              \@width\columnseprule}\hss}%
%    \end{macrocode}
%    As you will have noticed, we started with box register
%    |\mult@gfirstbox|  (i.e.\
%    the left column). So this time |\count@| looped through 2,
%    4,\ldots\ (plus the appropriate offset).
%    Finally we add box |\mult@rightbox| and we are done.
% \changes{v1.5a}{1992/11/04}{New box mechanism}
% \changes{v1.6a}{2003/03/15}{Preparing for adjusting \cs{prevdepth}}
%    Again we may have to update |\dimen\tw@|.
%    \begin{macrocode}
     \ifdim\dp\mult@rightbox>\dimen\tw@
       \global\dimen\tw@\dp\mult@rightbox \fi
%    \end{macrocode}
%    If the \texttt{colaction} option is given we write out status
%    information about the last column, otherwise the next command
%    does nothing.
% \changes{v1.8e}{2014/04/23}{Support \cs{docolaction}}
%    \begin{macrocode}
     \mc@lastcol@status@write
     \box\mult@rightbox
}
%    \end{macrocode}
%  \end{macro}

%  \begin{macro}{\RL@column@boxes}
%    Assembling the boxes for right to left typesetting is far more
%    complicated. When I first tried to build a solution for this my
%    thinking was that all that is necessary to do is to reverse the
%    order of the columns. But such an approach produces a subtle bug:
%    If we work this way then the first column put on the page will be
%    the last column of the text to read. and this means that the
%    order in which \TeX{} executes write statements or assembles mark
%    material will not happen in the order of the textual flow. So if,
%    for example each column contains a section command then these
%    sections will appear in reverse order in the table of content.
%
%    For this reason some amount of gymnastics is needed to add the
%    columns in their natural flow.
%    \begin{macrocode}
\def\RL@column@boxes{%
%    \end{macrocode}
%    First step is to put all rules in the right place (without adding
%    the comes which are instead represented by a space of |\hsize|.
%    \begin{macrocode}
     \process@cols\mult@gfirstbox{%
       \hskip\hsize
       \hss{\columnseprulecolor\vrule
              \@width\columnseprule}\hss
     }%
     \hskip\hsize
%    \end{macrocode}
%    At this point in the code our typesetting reference point is at
%    the right end of the rightmost column (or rather where that column
%    should appear).
%
%    We are now typesetting all columns by first backing up by their
%    width (which is |\hsize|) then typesetting the box and then
%    backing up again, but this time further, i.e., also across the
%    column separation. That will then enable us to typeset the next
%    column using the same approach until we are done with all but the
%    final column.
%    \begin{macrocode}
     \process@cols\mult@gfirstbox{%
       \ifdim\dp\count@>\dimen\tw@
         \global\dimen\tw@\dp\count@ \fi
       \hskip-\hsize
%    \end{macrocode}
%
% \changes{v1.8e}{2014/04/23}{Support \cs{docolaction}}
%    \begin{macrocode}
       \mc@col@status@write
       \box\count@
       \hskip-\hsize
       \hskip-\columnsep
     }%
%    \end{macrocode}
%    The approach for the final column is similar only that we do not
%    have to back up over any column gap.
%    \begin{macrocode}
     \ifdim\dp\mult@rightbox>\dimen\tw@
       \global\dimen\tw@\dp\mult@rightbox \fi
     \hskip-\hsize
%    \end{macrocode}
%
% \changes{v1.8e}{2014/04/23}{Support \cs{docolaction}}
%    \begin{macrocode}
     \mc@lastcol@status@write
     \box\mult@rightbox
     \hskip-\hsize
%    \end{macrocode}
%    However we do have to move the reference point to its right
%    place: to make the rules appear at the expected places, we should
%    get the typesetting position to the far right again. As we at the
%    moment at the far left we skip to the far right like this:
%    \begin{macrocode}
     \hskip\full@width
 }
%    \end{macrocode}
%  \end{macro}
%
%  \begin{macro}{\RLmulticolcolumns}
%  \begin{macro}{\LRmulticolcolumns}
%  \begin{macro}{\mc@align@columns}
%    Macros to switch between left-right and right-left typesetting. In LR
%    typesetting the |\LR@column@boxes| is used to combine
%    the columns. When typesetting right to left the |\RL@column@boxes|
%    is used instead.
%    \begin{macrocode}
\newcommand\RLmulticolcolumns
    {\let\mc@align@columns
         \RL@column@boxes}
\newcommand\LRmulticolcolumns
    {\let\mc@align@columns
      \LR@column@boxes}
%    \end{macrocode}
%    The default is left-to-right:
%    \begin{macrocode}
\LRmulticolcolumns
%    \end{macrocode}
%  \end{macro}
%  \end{macro}
%  \end{macro}
%
%
% \subsection{Supporting \texttt{\textbackslash docolaction}}
%
% \changes{v1.8e}{2014/04/23}{Support \cs{docolaction}}
%
%   Whenever we want to do something that depends on the current
%   column we execute \verb=\docolaction=. This command takes one
%   optional and three mandatory arguments. The mandatory ones denote
%   what to do if this is a ``left'', ``middle'', or ``right'' column
%   and the optional one is simply there to say what to do if we don't
%   know (default is to use the ``left'' column action in that case).
%
%  \begin{macro}{\mc@col@check@num}
%   We use one counter \verb=\mc@col@check@num= to generate us unique
%   label names. Each time we execute \verb=\docolaction= we increment
%   this counter to get a new name.
%    \begin{macrocode}
\newcount\mc@col@check@num
%    \end{macrocode}
%  \end{macro}
%
%   The generated ``labels'' are named
%\begin{verbatim}
%\mc@col-\the\mc@col@check@num
%\end{verbatim}
%   and they hold as values the
%   numbers 1, 2, or 3 denoting the current column type.

%  \begin{macro}{\docolaction}
%
%    \begin{macrocode}
\newcommand\docolaction[4][1]{%
 \ifx\mc@col@status@write\relax
     \PackageError{multicol}%
      {Option 'colaction' not selected}%
      {\string\docolaction\space
       requires the use of the 'colaction'
       option on the package}%
 \fi
 \global\advance\mc@col@check@num\@ne
 \edef\mc@col@type{\expandafter\ifx
   \csname mc@col-\the\mc@col@check@num
   \endcsname\relax
                   0\else
   \csname mc@col-\the\mc@col@check@num
   \endcsname
                    \fi}%
%    \end{macrocode}
%    We prefix with 0 so that an unknown label (that returns
%   \verb=\relax=) will result in case 0
%    \begin{macrocode}
 \ifcase \mc@col@type\relax
%    \end{macrocode}
%    If column is unknown we use the default action or the action
%   denoted by the optional argument (so that arg can take the value
%   1, 2, 3).
%    \begin{macrocode}
     \ifcase #1\or #2\or#3\or#4\fi
  \or
%    \end{macrocode}
%    Otherwise we know (or think we know) that this is a first, middle,
%   or last column:
%    \begin{macrocode}
     #2%  % 1 First col
  \or
     #3%  % 2 any middle col
  \or
     #4%  % 3 last col
  \else
    \ERROR
  \fi
%    \end{macrocode}
%    But how does the column number get associated with our label? We
%   do do this by writing another line into the aux file at this point:
%    \begin{macrocode}
  \edef\next{\write\@auxout
     {\string\mc@set@col@status
      {mc@col-\the\mc@col@check@num}%
      {\mc@col@type}}}%
  \next
}
%    \end{macrocode}
%  \end{macro}
%
%   Because of extra data writing to the aux file the aux file will
%   now contain something like the following after the document is
%   processed the first time:
%\begin{verbatim}
%\relax
%\mc@col@status{1}
%\mc@set@col@status{lcol-1}{0}
%\mc@col@status{2}
%\mc@set@col@status{lcol-2}{0}
%\mc@col@status{3}
%\mc@set@col@status{lcol-3}{0}
%\mc@col@status{1}
%\mc@col@status{2}
%\mc@col@status{3}
%\mc@set@col@status{lcol-4}{0}
%\end{verbatim}
%   The \verb=\mc@col@status= line denotes the column type and has been
%   written out just before corresponding the column box was placed
%   onto the page.
%   The\verb=\mc@set@col@status= lines have been written out as part
%   of shipping the column boxes out, e.g.,
%   \verb=\mc@set@col@status{lcol-1}{0}= was therefore somewhere within
%   the first column as it appears between \verb=\mc@col@status{1}=
%   and  \verb=\mc@col@status{2}=
%   The second argument in that line is the value used in the previous
%   run (or zero if there was no previous run. We can use this to
%   determine if a rerun is necessary.
%
%   Thus with this knowledge we can set things up to get the labels
%   working.
%
%
%  \begin{macro}{\mc@col@status}
%
%   When the aux file is read in \verb=\mc@col@status= is used to set
%   \verb=\mc@curr@col@status=:
%    \begin{macrocode}
\def\mc@col@status#1{%
    \gdef\mc@curr@col@status{#1}}
%    \end{macrocode}
%  \end{macro}
%
%
%
%  \begin{macro}{\mc@set@col@status}
%
%   And when \verb=\mc@set@col@status= is executed we can simply set
%   up the label by associating it with the \verb=\mc@curr@col@status=
%   and ignore the second argument:
%    \begin{macrocode}
\def\mc@set@col@status#1#2{%
  \global\expandafter\let\csname #1\endcsname
                         \mc@curr@col@status}
%    \end{macrocode}
%   The above definition is being used when the \texttt{.aux} file is
%   read in at the beginning. At the end we need a different
%   definition to test if another typesetting run is needed. There we
%   compare the value used in the current run (stored in the second
%   argument) with the value used on the next run. If those two values
%   differ we set \verb=@tempswa= to false which will trigger the
%   ``Label(s) may have changed'' warning.
%    \begin{macrocode}
\AtEndDocument{\def\mc@set@col@status#1#2{%
     \ifnum #2=\mc@curr@col@status\else
       \@tempswatrue
     \fi}%
}
%    \end{macrocode}
%  \end{macro}
%
%
%  \begin{macro}{mc@firstcol}
%   Finally, as part of determining in which column we are, we used a
%   switch inside |\mc@col@status@write| to determine if we are in the
%   first column or not.
%  \begin{macrocode}
\newif\ifmc@firstcol
\mc@firstcoltrue
%    \end{macrocode}
%  \end{macro}
% \end{multicols*}
%
% \Finale
%
\endinput


